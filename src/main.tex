%%%%%%%%%%%%%%%%%%%%%%%%%%%%%%%%%%%%%%%%%%%%%%%%%%%%%%%%%%%%%%%%%%%%%%%%%%%%%%%%%%%%%%
% Template fuer Abschlussarbeiten von Studierenden der 
% Frankfurt University of Applied Sciences
% 
% erstellt von: Prof. Dr.-Ing. Thomas Hollstein
% 
% Last revision: 22.06.2023
%
%%%%%%%%%%%%%%%%%%%%%%%%%%%%%%%%%%%%%%%%%%%%%%%%%%%%%%%%%%%%%%%%%%%%%%%%%%%%%%%%%%%%%%


\documentclass[
    %twoside, 
    %openright,
  titlepage,
  numbers=noenddot,
  headinclude,
    %1headlines,
  footinclude=true,
    %cleardoublepage=empty,
    %BCOR=5mm,
  fontsize=12pt,%11pt,
  paper=a4,
    %letterpaper
    %a4paper,
  ngerman,
  american,
    %table
  ]
% Dokumententyp: (legt grundlegende Formatierrichtlinien fest)
    %{book}
    %{report}
{scrreprt}
    %{scrreprt}

%%%%%%%%%%%%%%%%%%%%%%%%%%%%%%%%%%%%%%%%%%%%%%%%%%%%%%%%%%%%%%%%%%%%%%%%%%%%%%%%%%%%%%


   

% Farbigen Text
\usepackage{xcolor}

\parindent = 0pt  % Neue Absaetze nicht einruecken
\parskip = 1.0\baselineskip % paragraph gap: one baseline for clearer separation

% Improve overall micro-typography and line spacing for better readability
\usepackage{microtype} % better justification and protrusion
\usepackage{setspace}
% Use a slightly increased line spacing similar to the black example
% 1.2-1.25 gives a more open, readable look; choose 1.22 as compromise
\setstretch{1.22}





%%%%%%%%%% zu nutzende Pakete einbinden: %%%%%%%%%%

%%%%% Package Kommentare:



\usepackage{comment}
\usepackage[colorinlistoftodos]{todonotes}
\newcommand{\mycomment}[1]{\todo[inline,linecolor=green,backgroundcolor=yellow!25,bordercolor=green,caption={}]{todo: #1}}



%%%%% package Sprachunterstuetzung:

\usepackage{babel}[ngerman]
\usepackage{csquotes}

%%%%% Package Multirow ermoeglicht, dass sich ein Kaestchen in einer Tabelle ueber mehrere Zeilen erstrecken kann:

\usepackage{multirow}

% Beispiele unter: https://texblog.org/2012/12/21/multi-column-and-multi-row-cells-in-latex-tables/

%Vereinigung von Feldern in Tabellen:
%\multicolumn{number cols}{align}{text} % align: l,c,r
%\multirow{number rows}{width}{text}

%%%% Serifenloser Textstil fuer das ganze Dokument: 

% Use fontspec for custom fonts (requires XeLaTeX or LuaLaTeX)
\usepackage{fontspec}

% Set Mona Sans as the main/sans-serif font
\setmainfont{MonaSans}[
    Path = ../fonts/mona-sans/static/otf/,
    Extension = .otf,
    UprightFont = *-Regular,
    BoldFont = *-Bold,
    ItalicFont = *-Italic,
    BoldItalicFont = *-BoldItalic
]

% Set Mona Sans as sans-serif font explicitly
\setsansfont{MonaSans}[
    Path = ../fonts/mona-sans/static/otf/,
    Extension = .otf,
    UprightFont = *-Regular,
    BoldFont = *-Bold,
    ItalicFont = *-Italic,
    BoldItalicFont = *-BoldItalic
]

% Set Monaspace for monospaced/code text
\setmonofont{Monaspace Neon}[
    Path = ../fonts/mona-space/Monaspace Neon/,
    Extension = .ttf,
    UprightFont = * Var
]

% Base font of the document is to be sans serif
\renewcommand*\familydefault{\sfdefault}

%%%%%% Grafiken einbinden
\usepackage{graphicx}
% https://golatex.de//wiki/%5cincludegraphics

%%%%%% Caption styling - even smaller and lighter
\usepackage{caption}
% Macro to select punctuation after the label: colon or period
% Usage: set to ":" or "." in the preamble
\providecommand{\CaptionLabelPunct}{.}
% Define a custom label format that appends the chosen punctuation
\DeclareCaptionLabelFormat{labelsepWithPunct}{#1~#2\CaptionLabelPunct\,}
\captionsetup{
  font={footnotesize,it},         % smaller and italic font for the caption
  labelfont={footnotesize,it},    % label smaller and italic (no bold)
  textfont={footnotesize,it},     % caption text smaller and italic
  labelsep=none,                 % we'll use a custom label format
  labelformat=labelsepWithPunct, % use the custom label format defined above
  justification=centering,       % Center-aligned captions
  skip=6pt                       % Reduced space between figure and caption
}

%%%%%% Mathematik-Paket AMSMath:
\usepackage[fleqn,reqno]{amsmath}
\usepackage{amssymb}  % Additional math symbols (e.g., \mathbb for number sets)


%%%%%% Seitengeometrie einstellen:
% https://tex.stackexchange.com/questions/344241/logo-as-header-using-fancyhdr-package
\usepackage{geometry}
\geometry{verbose,
          bmargin=2.5cm,
          lmargin=2cm,
          rmargin=2cm
          %footskip=-25pt
          }

%%%%%% die Höhe des Top-Margin errechnen:
\newlength\mytopmargin
\newsavebox{\headbox}\savebox{\headbox}{
    % Smaller header box for more compact design
    \raisebox{-0.5ex}{\includegraphics[height=1.2cm]{Figures/vgu-logo.pdf}}
}
\setlength{\mytopmargin}{\totalheightof{\usebox{\headbox}}+1.5cm}
\geometry{verbose,
          tmargin=\mytopmargin,
          headheight=1.8cm,
          footskip=8ex
}

% Flexiblere Tabellen
\usepackage{tabularx}
\def\tabularxcolumn#1{m{#1}}

% Boxen für Lickert Skalen
\usepackage{wasysym}
\newcommand\insq[1]{%
    \Square\ #1\quad%
}

% Durchstreichen von Text
\usepackage[normalem]{ulem} % mit \sout

%%%%%% Code listings with syntax highlighting
\usepackage{listings}
\usepackage{xcolor} % already loaded earlier, but needed for listings colors
\lstset{
    basicstyle=\ttfamily\small,
    keywordstyle=\color{blue}\bfseries,
    commentstyle=\color{gray}\itshape,
    stringstyle=\color{red},
    numberstyle=\tiny\color{gray},
    numbers=left,
    numbersep=8pt,
    frame=single,
    breaklines=true,
    breakatwhitespace=true,
    showstringspaces=false,
    tabsize=2,
    captionpos=b,
    aboveskip=1.5em,
    belowskip=1em
}

%%%%%% Algorithm and pseudocode formatting
\usepackage{algorithm}
\usepackage{algpseudocode}

%%%%%% Subfigures support
\usepackage{subcaption}

%%%%%% Professional table formatting (booktabs)
\usepackage{booktabs}

%%%%%% Enhanced list environments - using paralist (already loaded earlier)
% Note: enumitem conflicts with paralist, so we use paralist which is already loaded



%%%%%% Gesamtseitenzahl verwenden:
\usepackage{totpages}

%%%%%% Kalkulationen:
%https://tex.stackexchange.com/questions/30081/how-can-i-sum-two-values-and-store-the-result-in-other-variable
\usepackage{tikz}
\usetikzlibrary{math}

%%%%%% Gestaltung von Kopf- und Fusszeilen:
% https://tex.stackexchange.com/questions/344241/logo-as-header-using-fancyhdr-package
\usepackage{fancyhdr}
\pagestyle{fancy}  % Eigener Seitenstil
\fancyhf{}         % Alle Kopf- und Fußzeilenfelder bereinigen

% Header configuration - smaller and more elegant
\fancyhead[L]{%
  \fontsize{9}{11}\selectfont\fontseries{l}\selectfont% Light weight, smaller font
  \leftmark%
}
\fancyhead[C]{%
  \fontsize{8.5}{10}\selectfont\fontseries{l}\selectfont% Extra light, smaller
  \rightmark%
}
\fancyhead[R]{%
  \raisebox{-0.5ex}{% Reduced vertical offset
    \includegraphics[height=0.5cm]{Figures/vgu-logo.pdf}% Smaller logos
    \hspace{0.5em}% Reduced spacing
    \includegraphics[height=0.5cm]{Figures/company-logo.png}%
  }%
}

\renewcommand{\headrulewidth}{0.3pt} % Thinner header line
\setlength{\headheight}{1.8cm} % Reduced header height

% Footer configuration - minimal and elegant
\fancyfoot[C]{%
  \fontsize{8.5}{10}\selectfont\fontseries{l}\selectfont% Light weight
  \textit{\ThesisTitleShort}% Italic for elegance
} 
\fancyfoot[R]{%
  \fontsize{9}{11}\selectfont\fontseries{m}\selectfont% Medium weight
  \thepage% Page number
}
\renewcommand{\footrulewidth}{0.2pt} % Thinner footer line
\setlength{\mytopmargin}{\totalheightof{\usebox\headbox} +1.5cm}% Reduced margin
%Unterschied zwischen geraden/ungeraden Seiten:
%\fancyhead[OR]{} % "O" steht für "odd", also ungerade Seiten
%\fancyhead[ER]{} % "E" für "even", also gerade Seiten.




%%%%% Erweiterte Formate für Listen/Aufzaehlungen:
\usepackage{paralist}
%Default-Items fuer die vier moeglichen Verschachtelungsebenen:
\setdefaultitem{}{\textbullet}{$\star$}{}

%%%%% FRA-UAS CI Farben:
%\definecolor{airforceblue}{rgb}{0.36, 0.54, 0.66}
% CI-Farben FRA-UAS (blau):
\definecolor{FRAUAS_Blue_Dark}{RGB}{45, 137, 204}
\definecolor{FRAUAS_Blue_Light}{RGB}{182, 210, 228}
% CI-Farben FRA-UAS: FB1
\definecolor{FRAUAS_FB1_Dark}{RGB}{124, 128, 52}
\definecolor{FRAUAS_FB1_Light}{RGB}{213, 213, 179}
% CI-Farben FRA-UAS: FB2
\definecolor{FRAUAS_FB2_Dark}{RGB}{255, 158, 27}
\definecolor{FRAUAS_FB2_Light}{RGB}{251, 221, 173}
% CI-Farben FRA-UAS: FB3
\definecolor{FRAUAS_FB3_Dark}{RGB}{196, 213, 42}
\definecolor{FRAUAS_FB3_Light}{RGB}{237, 240, 166}
% CI-Farben FRA-UAS: FB4
\definecolor{FRAUAS_FB4_Dark}{RGB}{204, 31, 47}
\definecolor{FRAUAS_FB4_Light}{RGB}{240, 166, 183}

%% VGU COLOR
\definecolor{VGU_ORANGE}{RGB}{196, 67, 45}

%%%%% Sektionstitel nach CI-Farben einfärben:
\usepackage{titlesec}
\titleformat{\section}
{\color{VGU_ORANGE}\normalfont\Large\bfseries} %Titel
{\color{VGU_ORANGE}\thesection}{1em}{}
\titleformat{\subsection}
{\color{VGU_ORANGE}\normalfont\large\bfseries} %Titel
{\color{VGU_ORANGE}\thesubsection}{1em}{}

%\usepackage{appendix}

%%%%% Erweiterte Bibliographie-Stile:
%\usepackage{harvard}
% Title Page
% (wir generieren den Titel per Handlayout und verwenden
% daher die folgenden Befehle nicht)
%\title{}
%\author{}

% https://www.overleaf.com/learn/latex/Hyperlinks
\usepackage{hyperref}
\hypersetup{
    %hyperindex=true,
    %linktocpage=true, % Seitenzahl statt Titel verlinkt
    colorlinks=true,
    linkcolor=black,
    filecolor=black,
    urlcolor=black,
    pdftitle={Thesis},
    pdfpagemode=FullScreen,
    citecolor=black,
    }



%\usepackage[hyphens]{url}  %% stellt \url{} zur Verfuegung

%%%%% modernes BibLaTeX mit biber %%%%%
% https://golatex.de/viewtopic.php?t=13917
%\usepackage[style=ieee-alphabetic,
%backend=biber, natbib=true]{biblatex}

% Literatur nach Erscheinen im Text sortiert
%\usepackage[sorting=none, style=numeric,backend=biber, natbib=true]{biblatex}
%\usepackage[style=apa, backend=biber, natbib=true, sorting=nyt, sortcites=false]{biblatex}
\usepackage[style=numeric, backend=biber, natbib=true, sorting=nyt, sortcites=false]{biblatex}
% Ohne eingestellte Sortierung
%\usepackage[ style=numeric,
%backend=biber, natbib=true]{biblatex}


\addbibresource{bibliography.bib}

% Immer shortautor anzeigen, falls vorhanden
\makeatletter
\def\cbx@apa@ifnamesaved{\@firstoftwo}
\makeatother

% Zeilenabstand für das ganze Dokument (1.0 = Normalwert):
\renewcommand{\baselinestretch}{1.0}

%%%%% Glossaries and Acronyms setup:
\usepackage[acronym,toc]{glossaries}
\makeglossaries

%%%%% Acronymdefinitionen einbinden:
%%%%%%% Definition Akronyme:

% AI/ML Related
\newacronym{ai}{AI}{Artificial Intelligence}
\newacronym{llm}{LLM}{Large Language Model}
\newacronym{rag}{RAG}{Retrieval-Augmented Generation}
\newacronym{ml}{ML}{Machine Learning}
\newacronym{nlp}{NLP}{Natural Language Processing}

% Architecture/System Related
\newacronym{api}{API}{Application Programming Interface}
\newacronym{erp}{ERP}{Enterprise Resource Planning}
\newacronym{cicd}{CI/CD}{Continuous Integration/Continuous Deployment}
\newacronym{gcp}{GCP}{Google Cloud Platform}
\newacronym{hitl}{HITL}{Human-in-the-Loop}
\newacronym{adr}{ADR}{Architecture Decision Record}

% Quality/Testing Related
\newacronym{qa}{QA}{Quality Assurance}
\newacronym{nfr}{NFR}{Non-Functional Requirement}
\newacronym{dlq}{DLQ}{Dead Letter Queue}

% Other Technical Terms
\newacronym{sdk}{SDK}{Software Development Kit}
\newacronym{ui}{UI}{User Interface}
\newacronym{ux}{UX}{User Experience}
\newacronym{pm}{PM}{Product Manager}
\newacronym{pe}{PE}{Product Engineer}


%%%%%%% Definition Glossareinträge:

\newglossaryentry{langgraph}
{
    name=LangGraph,
    description={A library for building stateful, multi-actor applications with LLMs, built on top of LangChain}
}

\newglossaryentry{langchain}
{
    name=LangChain,
    description={A framework for developing applications powered by language models}
}

\newglossaryentry{langfuse}
{
    name=Langfuse,
    description={An open-source observability platform for LLM applications}
}

\newglossaryentry{n8n}
{
    name=n8n,
    description={An open-source workflow automation tool for connecting various services}
}

\newglossaryentry{cloudrun}
{
    name=Cloud Run,
    description={A fully managed compute platform by Google Cloud for running containerized applications}
}

\newglossaryentry{redis}
{
    name=Redis,
    description={An in-memory data structure store used as a database, cache, and message broker}
}

\newglossaryentry{postgres}
{
    name=PostgreSQL,
    description={An open-source relational database management system}
}

\newglossaryentry{hydra}
{
    name=Hydra,
    description={A framework for elegantly configuring complex applications}
}




%%%%%%%%%%%%%%%%%%%%%%%%%%%%%%%%%%%%%%%%%%%%
%%%%% hier beginnt das eigentliche Dokument 
%%%%%%%%%%%%%%%%%%%%%%%%%%%%%%%%%%%%%%%%%%%%

\begin{document}


%%%%% Settings einbinden:
\newcommand{\myName}{Tran Hai Duong}
\newcommand{\ThesisTitle}{Practical Training Report}
\newcommand{\ThesisTitleShort}{Practical Training Report - AI Agent Research and Development}
\newcommand{\ThesisSubtitle}{AI Agent Research and Development}
\newcommand{\ThesisDegree}{Bachelor of Science (B.Sc.)}
%\newcommand{\ThesisDegree}{Bachelor of Arts (B.A.)}
%\newcommand{\ThesisDegree}{Master of Science (M.Sc.)}
%\newcommand{\ThesisDegree}{Master of Arts (M.A.)}
\newcommand{\myStudentId}{10422021}
\newcommand{\Supervisor}{Mr. Luong Phuc Thien}
\newcommand{\CoSupervisor}{M.Sc. Donald Duck}
\newcommand{\Faculty}{Faculty of Engineering}
\newcommand{\University}{Vietnamese-German University}
\newcommand{\UniversityLocation}{Ho Chi Minh City}
\newcommand{\ThesisDeliveryDate}{31. December 2025}
\newcommand{\CompanyName}{DevSamurai Vietnam Joint Stock Company}  % don't modify this line, if no company is involved
% Additional fields used by the Practical Training Report titlepage
\newcommand{\InternPeriod}{04.09.2025 -- 31.12.2025}% e.g. 01.06.2025 -- 31.08.2025
\newcommand{\Intake}{2022}% e.g. Fall 2021
\newcommand{\ReportYear}{2025}


%%%%%%%%%%%%%%%%%%%%% FOR ENGLISH LANGUAGE THESIS: %%%%%%%%%%%%%%%%%%%%

% By default the thesis language is german
% if you want to set it to ENGLISH, then UNCOMMENT the FOLLOWING LINE by removing the leading "%":

\newcommand*{\ThesisLanguageIsEnglish}{}


\sloppy %Formatierungsueberstaende am Zeilenende vermeiden
% https://latexref.xyz/_005cfussy-_0026-_005csloppy.html

\frenchspacing
%Ein Leerzeichen nach Satzende
%https://texwelt.de/fragen/1154/was-ist-french-spacing-was-macht-frenchspacing

%\raggedbottom   
%Standard is \flushbottom, dass heisst
%alle Seiten werden so gedehnt, dass sie 
%gleich hoch sind 
%Schaltet man \raggedbottom ein, ist dies
%nicht so

\ifdefined\ThesisLanguageIsEnglish
  \selectlanguage{american}
\else
  \selectlanguage{ngerman} % ngerman, american
\fi
%Deutsch nach neuer Rechtschreibung als
%Standardsprache fuer das Dokument einstellen

%\renewcommand*{\bibname}{new name}
%\setbibpreamble{}

% Numerierungstiefe setzen:
% 2: bis subsection (Standard)
% 3: bis subsubsection
\setcounter{secnumdepth}{3} %setzt die Numerierungstiefe



%\renewcommand{\thepage}{\Roman{page}}
\pagestyle{plain} % Seite ohne Kopf- und Fusszeilen darstellen



% hier wird die Titelseite eingebunden:
\thispagestyle{empty}
%\pdfbookmark[0]{Titelblatt}{title}

%*******************************************************
% Titlepage
%*******************************************************
%%%
%%% title page (german)
%%%
\thispagestyle{empty}
\pdfbookmark[0]{Titelblatt}{title}
\begin{titlepage}


  \vspace*{-2.5cm}
  \begin{center}
    \begin{minipage}[t]{0.40\textwidth}
      \raggedright
      \includegraphics[width=4.8cm,keepaspectratio]{Figures/vgu-logo}
    \end{minipage}\hfill
    \begin{minipage}[t]{0.40\textwidth}
      \raggedleft
      \includegraphics[width=5.3cm,keepaspectratio]{Figures/company-logo}
    \end{minipage}
  \end{center}

  \begin{center}
    \vspace{0.1cm}
    \LARGE \textbf{Vietnamese-German University}\\
    \vspace{0.4cm}
    \Large -- Faculty of Engineering --\\
    \vspace{0.1cm}
    \large Computer Science Program
  \end{center}

  \vfill

  \begin{center}
    \Huge \textbf{\ThesisTitle}\\   %%%%% >>>>> Bitte in TeXFiles/000_Settings eintragen 
    % \huge {\ThesisSubtitle}
  \end{center}

  \vfill

  % Practical Training Report layout (English)
  \ifdefined\ThesisLanguageIsEnglish
    \begin{center}
      \vspace{0.3cm}
      {\large
        \setlength{\tabcolsep}{12pt}%
        \renewcommand{\arraystretch}{1.4}%
        \begin{tabular}{@{}ll}
          Company:       & \textbf{\CompanyName} \\
          Intern period: & \InternPeriod         \\
          Supervisor:    & \Supervisor           \\
          \\
          Author:        & \textbf{\myName}      \\
          Program:       & Computer Science      \\
          Student ID:    & \myStudentId          \\
          Intake:        & \Intake               \\
        \end{tabular}
      }
      \vspace{3cm}
      \\
      {\large \ReportYear, Vietnam}
    \end{center}
  \else
    \begin{center}
      \Large Abschlussarbeit zur Erlangung des akademischen Grades\\
      \vspace{0.3cm}
      \Large \ThesisDegree   %%%%% >>>>> Bitte in TeXFiles/000_Settings eintragen 
    \end{center}
  \fi

\end{titlepage}

\input{TeXFiles/002_NDNotice}
%*******************************************************
% Declaration
%*******************************************************

\ifdefined\ThesisLanguageIsEnglish
    \chapter*{Declaration}
    \thispagestyle{empty}

    Hereby I assure that I have written the presented practical training report independently and without any third party help and that I have not used any other tools or resources than the ones mentioned/referred to in the report.
    \medskip

    \noindent
    Any parts of the report that have been taken from other published or yet unpublished works in terms of their wording or meaning are thoroughly marked with an indication of the source.
    \medskip

    \noindent
    All figures in this report have been drawn by myself or are clearly attributed with a reference.
    \medskip

    \noindent
    This work has not been published or presented to any other examination authority before.
    I am aware of the importance of the affidavit and the consequences under examination
    law as well as the criminal consequences of an incorrect or incomplete affidavit.
    \medskip


\else
    \chapter*{Eidesstattliche Erklärung}
    \thispagestyle{empty}
    Ich versichere hiermit, dass ich die vorliegende Arbeit selbständig verfasst und keine anderen als die im Literaturverzeichnis angegebenen Quellen benutzt habe.
    \medskip

    \noindent
    Alle Stellen, die wörtlich oder sinngemäß aus veröffentlichten oder noch nicht veröffentlichten Quellen entnommen sind, sind als solche kenntlich gemacht.
    \medskip

    \noindent
    Die Zeichnungen oder Abbildungen in dieser Arbeit sind von mir selbst erstellt worden oder mit einem entsprechenden Quellennachweis versehen.
    \medskip

    \noindent
    Diese Arbeit ist in gleicher oder ähnlicher Form noch bei keiner anderen Prüfungsbehörde eingereicht worden.
    \bigskip
\fi

Ho Chi Minh City, \ThesisDeliveryDate

\smallskip


\hfill \makebox[10cm]{ } \\
\vspace*{-1ex}
\begin{flushright}
    \begin{tabular}{m{5cm}}
        \hline
        \vspace{0.3cm}
        \centering \textbf{\myName} \\
    \end{tabular}
\end{flushright}
%\textbf{\myName}
\newpage

% da wir den Titel haendisch erstellt haben entfaellt der folgende Befehl:
%\maketitle

%\clearpage

%\pagestyle{headings} 
\pagestyle{fancy}   % Seite mit Kopf- und Fusszeilen darstellen
\pagenumbering{roman}  % Umschalten auf Seitenzahlen in römischer Darstellung
%\fancyfoot[R]{i}  % Seitennummer
%roemisch 1, hier per Hand eingetragen, weil es anders nicht funktioniert hat - hat sich erledigt


\tableofcontents
\clearpage
\pagenumbering{arabic}  % Umschalten auf Seitenzahlen in arabischer Darstellung
%%%%% Einbinden der Dateien für die einzelnen Sektionen:
% Verwendet man hier \include statt \input, beginnen
% neue Sektionen immer auf einer neuen Seite
%\input{TeXFiles/Abstract}

\fancyfoot[R]{\thepage}  % Seitennummer


%\myName
\input{TeXFiles/010_Kurzfassung}
\chapter*{\LaTeX{} How-To Guide}
\label{ch:LaTeXHowTo}
\addcontentsline{toc}{chapter}{\LaTeX{} How-To Guide}

This chapter provides a comprehensive guide on how to use \LaTeX{} for writing technical reports. It covers essential features including document structure, cross-referencing, figures, tables, mathematical equations, code listings, citations, and more.

%=============================================================================
\section{Document Structure and Cross-References}
\label{sec:HowTo_Structure}
%=============================================================================

\subsection{Chapters, Sections, and Subsections}
\label{sec:HowTo_Sections}

\LaTeX{} provides a hierarchical structure for organizing your document:

\begin{itemize}
  \item \texttt{\textbackslash chapter\{Title\}} -- Main divisions (in report/book classes)
  \item \texttt{\textbackslash section\{Title\}} -- Major sections within chapters
  \item \texttt{\textbackslash subsection\{Title\}} -- Subsections
  \item \texttt{\textbackslash subsubsection\{Title\}} -- Further subdivisions
  \item \texttt{\textbackslash paragraph\{Title\}} -- Named paragraphs
\end{itemize}

\subsection{Labels and Cross-References}
\label{sec:HowTo_CrossRef}

Every element that you want to reference should have a \texttt{\textbackslash label\{key\}} command. You can then reference it using:

\begin{itemize}
  \item \texttt{\textbackslash ref\{key\}} -- Returns the number (e.g., ``Section~\ref{sec:HowTo_CrossRef}'')
  \item \texttt{\textbackslash pageref\{key\}} -- Returns the page number
  \item \texttt{\textbackslash autoref\{key\}} -- Automatically includes the type (e.g., ``Section'')
\end{itemize}

For example, we are currently in Section~\ref{sec:HowTo_Structure}, specifically in Subsection~\ref{sec:HowTo_CrossRef}.

%=============================================================================
\section{Figures and Graphics}
\label{sec:HowTo_Figures}
%=============================================================================

\subsection{Basic Figure Insertion}
\label{sec:HowTo_BasicFigure}

Figures are inserted using the \texttt{figure} environment with the \texttt{graphicx} package:

\begin{figure}[htbp]
  \centering
  \includegraphics[width=0.5\textwidth]{Figures/vgu-logo.pdf}
  \caption{Example of a centered figure with caption.}
  \label{fig:ExampleLogo}
\end{figure}

The placement specifiers \texttt{[htbp]} control where \LaTeX{} tries to place the figure:
\begin{itemize}
  \item \texttt{h} -- Here (approximately where it appears in the source)
  \item \texttt{t} -- Top of the page
  \item \texttt{b} -- Bottom of the page
  \item \texttt{p} -- On a separate page of floats
  \item \texttt{!} -- Override internal parameters
\end{itemize}

To reference Figure~\ref{fig:ExampleLogo}, use \texttt{\textbackslash ref\{fig:ExampleLogo\}}.

\subsection{Subfigures}
\label{sec:HowTo_Subfigures}

For multiple related figures, use the \texttt{subcaption} package:

\begin{figure}[htbp]
  \centering
  \begin{subfigure}[b]{0.45\textwidth}
    \centering
    \includegraphics[width=0.8\textwidth]{Figures/vgu-logo.pdf}
    \caption{First subfigure.}
    \label{fig:sub1}
  \end{subfigure}
  \hfill
  \begin{subfigure}[b]{0.45\textwidth}
    \centering
    \includegraphics[width=0.8\textwidth]{Figures/company-logo.png}
    \caption{Second subfigure.}
    \label{fig:sub2}
  \end{subfigure}
  \caption{Example of subfigures: (a) VGU Logo, (b) Company Logo.}
  \label{fig:Subfigures}
\end{figure}

Reference individual subfigures as Figure~\ref{fig:sub1} or the whole figure as Figure~\ref{fig:Subfigures}.

%=============================================================================
\section{Tables}
\label{sec:HowTo_Tables}
%=============================================================================

\subsection{Basic Tables with Booktabs}
\label{sec:HowTo_BasicTables}

Professional tables use the \texttt{booktabs} package for clean horizontal rules:

\begin{table}[htbp]
  \centering
  \caption{Example of a professional table using booktabs.}
  \label{tab:BasicTable}
  \begin{tabular}{lcc}
    \toprule
    \textbf{Method}  & \textbf{Accuracy (\%)} & \textbf{Time (s)} \\
    \midrule
    Baseline         & 85.2                   & 1.23              \\
    Proposed         & 92.7                   & 0.89              \\
    State-of-the-art & 91.3                   & 2.15              \\
    \bottomrule
  \end{tabular}
\end{table}

Key commands from \texttt{booktabs}:
\begin{itemize}
  \item \texttt{\textbackslash toprule} -- Top rule (thicker)
  \item \texttt{\textbackslash midrule} -- Middle rule (thinner)
  \item \texttt{\textbackslash bottomrule} -- Bottom rule (thicker)
  \item \texttt{\textbackslash cmidrule\{i-j\}} -- Partial rule from column $i$ to $j$
\end{itemize}

\subsection{Multi-row and Multi-column Cells}
\label{sec:HowTo_MultiRowCol}

Use \texttt{multirow} and \texttt{multicolumn} for spanning cells:

\begin{table}[htbp]
  \centering
  \caption{Table with multi-row and multi-column cells.}
  \label{tab:MultiTable}
  \begin{tabular}{lccc}
    \toprule
                         & \multicolumn{3}{c}{\textbf{Metrics}}                                       \\
    \cmidrule(lr){2-4}
    \textbf{Model}       & \textbf{Precision}                   & \textbf{Recall} & \textbf{F1-Score} \\
    \midrule
    \multirow{2}{*}{CNN} & 0.89                                 & 0.87            & 0.88              \\
                         & 0.91                                 & 0.85            & 0.88              \\
    \midrule
    \multirow{2}{*}{RNN} & 0.85                                 & 0.90            & 0.87              \\
                         & 0.86                                 & 0.88            & 0.87              \\
    \bottomrule
  \end{tabular}
\end{table}

%=============================================================================
\section{Mathematical Equations}
\label{sec:HowTo_Math}
%=============================================================================

\subsection{Inline and Display Math}
\label{sec:HowTo_InlineMath}

Inline math is written between dollar signs: The equation $E = mc^2$ is famous. For display equations, use the \texttt{equation} environment:

\begin{equation}
  f(x) = \int_{-\infty}^{\infty} \hat{f}(\xi) e^{2\pi i \xi x} \, d\xi
  \label{eq:FourierTransform}
\end{equation}

Equation~\ref{eq:FourierTransform} shows the inverse Fourier transform.

\subsection{Multi-line Equations with Alignment}
\label{sec:HowTo_AlignedMath}

Use the \texttt{align} environment for aligned multi-line equations:

\begin{align}
  \nabla \cdot \mathbf{E}  & = \frac{\rho}{\varepsilon_0} \label{eq:Maxwell1}                                                    \\
  \nabla \cdot \mathbf{B}  & = 0 \label{eq:Maxwell2}                                                                             \\
  \nabla \times \mathbf{E} & = -\frac{\partial \mathbf{B}}{\partial t} \label{eq:Maxwell3}                                       \\
  \nabla \times \mathbf{B} & = \mu_0 \mathbf{J} + \mu_0 \varepsilon_0 \frac{\partial \mathbf{E}}{\partial t} \label{eq:Maxwell4}
\end{align}

These are Maxwell's equations (Equations~\ref{eq:Maxwell1}--\ref{eq:Maxwell4}).

\subsection{Matrices and Arrays}
\label{sec:HowTo_Matrices}

Various matrix environments are available:

\begin{equation}
  \mathbf{A} = \begin{bmatrix}
    a_{11} & a_{12} & a_{13} \\
    a_{21} & a_{22} & a_{23} \\
    a_{31} & a_{32} & a_{33}
  \end{bmatrix}
  \qquad
  \mathbf{B} = \begin{pmatrix}
    b_1 \\
    b_2 \\
    b_3
  \end{pmatrix}
  \label{eq:Matrices}
\end{equation}

Other matrix styles: \texttt{matrix} (no delimiters), \texttt{vmatrix} (vertical bars for determinants), \texttt{Vmatrix} (double vertical bars).

\subsection{Common Mathematical Symbols}
\label{sec:HowTo_MathSymbols}

\begin{itemize}
  \item Greek letters: $\alpha, \beta, \gamma, \delta, \epsilon, \theta, \lambda, \mu, \sigma, \phi, \omega$
  \item Operators: $\sum_{i=1}^{n}$, $\prod_{j=1}^{m}$, $\int_a^b$, $\lim_{x \to \infty}$
  \item Relations: $\leq, \geq, \neq, \approx, \equiv, \propto, \sim$
  \item Sets: $\in, \notin, \subset, \subseteq, \cup, \cap, \emptyset, \mathbb{R}, \mathbb{N}$
  \item Arrows: $\rightarrow, \leftarrow, \Rightarrow, \Leftrightarrow, \mapsto$
  \item Fractions: $\frac{a}{b}$, $\dfrac{dy}{dx}$
  \item Roots: $\sqrt{x}$, $\sqrt[n]{x}$
\end{itemize}

%=============================================================================
\section{Code Listings}
\label{sec:HowTo_Code}
%=============================================================================

\subsection{Basic Code Listing}
\label{sec:HowTo_BasicCode}

Use the \texttt{listings} package for syntax-highlighted code:

\begin{lstlisting}[language=Python, caption={Example Python function for calculating factorial.}, label={lst:PythonExample}]
def factorial(n):
    """Calculate the factorial of n recursively."""
    if n <= 1:
        return 1
    else:
        return n * factorial(n - 1)

# Example usage
result = factorial(5)  # Returns 120
print(f"5! = {result}")
\end{lstlisting}

Reference code listings as Listing~\ref{lst:PythonExample}.

\subsection{Different Programming Languages}
\label{sec:HowTo_CodeLanguages}

The \texttt{listings} package supports many languages:

\begin{lstlisting}[language=C, caption={Example C code for bubble sort.}, label={lst:CExample}]
void bubbleSort(int arr[], int n) {
    for (int i = 0; i < n - 1; i++) {
        for (int j = 0; j < n - i - 1; j++) {
            if (arr[j] > arr[j + 1]) {
                // Swap elements
                int temp = arr[j];
                arr[j] = arr[j + 1];
                arr[j + 1] = temp;
            }
        }
    }
}
\end{lstlisting}

\begin{lstlisting}[language=Java, caption={Example Java class definition.}, label={lst:JavaExample}]
public class Person {
    private String name;
    private int age;
    
    public Person(String name, int age) {
        this.name = name;
        this.age = age;
    }
    
    public String getName() {
        return this.name;
    }
}
\end{lstlisting}

%=============================================================================
\section{Algorithms and Pseudocode}
\label{sec:HowTo_Algorithms}
%=============================================================================

For presenting algorithms, use the \texttt{algorithm} and \texttt{algpseudocode} packages:

\begin{algorithm}[htbp]
  \caption{Binary Search Algorithm}
  \label{alg:BinarySearch}
  \begin{algorithmic}[1]
    \Require Sorted array $A[1..n]$, target value $x$
    \Ensure Index of $x$ in $A$, or $-1$ if not found
    \State $left \gets 1$
    \State $right \gets n$
    \While{$left \leq right$}
    \State $mid \gets \lfloor (left + right) / 2 \rfloor$
    \If{$A[mid] = x$}
    \State \Return $mid$
    \ElsIf{$A[mid] < x$}
    \State $left \gets mid + 1$
    \Else
    \State $right \gets mid - 1$
    \EndIf
    \EndWhile
    \State \Return $-1$
  \end{algorithmic}
\end{algorithm}

Algorithm~\ref{alg:BinarySearch} shows the classic binary search with $O(\log n)$ time complexity.

%=============================================================================
\section{Citations and Bibliography}
\label{sec:HowTo_Citations}
%=============================================================================

\subsection{Citation Commands}
\label{sec:HowTo_CitationCommands}

This template uses BibLaTeX with the \texttt{natbib} compatibility layer. Common citation commands:

\begin{itemize}
  \item \texttt{\textbackslash citep\{key\}} -- Parenthetical citation: \citep{friese_handbuch_2020}
  \item \texttt{\textbackslash citet\{key\}} -- Textual citation: \citet{friese_handbuch_2020}
  \item \texttt{\textbackslash cite\{key\}} -- Basic citation: \cite{friese_handbuch_2020}
\end{itemize}

For multiple citations: \citep{friese_handbuch_2020, beermann_veranderungen_2020}.

\subsection{Managing Bibliography}
\label{sec:HowTo_Bibliography}

Bibliography entries are stored in \texttt{bibliography.bib}. Common entry types:

\begin{itemize}
  \item \texttt{@article} -- Journal articles
  \item \texttt{@book} -- Books
  \item \texttt{@inproceedings} -- Conference papers
  \item \texttt{@techreport} -- Technical reports
  \item \texttt{@misc} -- Websites and other sources
  \item \texttt{@phdthesis}, \texttt{@mastersthesis} -- Theses
\end{itemize}

%=============================================================================
\section{Acronyms and Glossary}
\label{sec:HowTo_Acronyms}
%=============================================================================

The \texttt{glossaries} package manages acronyms and glossary entries.

\subsection{Using Acronyms}
\label{sec:HowTo_UsingAcronyms}

Acronym commands:
\begin{itemize}
  \item \texttt{\textbackslash acrshort\{key\}} -- Short form: \acrshort{gcd}
  \item \texttt{\textbackslash acrlong\{key\}} -- Long form: \acrlong{gcd}
  \item \texttt{\textbackslash acrfull\{key\}} -- Full form: \acrfull{lcm}
  \item \texttt{\textbackslash gls\{key\}} -- Smart reference (expands on first use)
\end{itemize}

The \acrfull{gcd} is used in number theory to find common factors. Later references use \acrshort{gcd}.

\subsection{Glossary Entries}
\label{sec:HowTo_Glossary}

Glossary entries are referenced with \texttt{\textbackslash gls\{key\}}, \texttt{\textbackslash Gls\{key\}} (capitalized), or \texttt{\textbackslash glspl\{key\}} (plural). For example: \Gls{latex} is a document preparation system based on \gls{maths}.

%=============================================================================
\section{Lists and Enumerations}
\label{sec:HowTo_Lists}
%=============================================================================

\subsection{Itemized Lists}
\label{sec:HowTo_Itemize}

\begin{itemize}
  \item First level item
  \item Another first level item
        \begin{itemize}
          \item Second level item
          \item Another second level item
                \begin{itemize}
                  \item Third level item
                \end{itemize}
        \end{itemize}
\end{itemize}

\subsection{Enumerated Lists}
\label{sec:HowTo_Enumerate}

\begin{enumerate}
  \item First step
  \item Second step
        \begin{enumerate}
          \item Sub-step 2.1
          \item Sub-step 2.2
        \end{enumerate}
  \item Third step
\end{enumerate}

\subsection{Description Lists}
\label{sec:HowTo_Description}

\begin{description}
  \item[Term 1] Definition of the first term.
  \item[Term 2] Definition of the second term with more detailed explanation.
  \item[Term 3] Brief definition.
\end{description}

%=============================================================================
\section{Text Formatting}
\label{sec:HowTo_Formatting}
%=============================================================================

\subsection{Basic Text Styles}
\label{sec:HowTo_TextStyles}

\begin{itemize}
  \item \textbf{Bold text} -- \texttt{\textbackslash textbf\{...\}}
  \item \textit{Italic text} -- \texttt{\textbackslash textit\{...\}}
  \item \underline{Underlined text} -- \texttt{\textbackslash underline\{...\}}
  \item \texttt{Monospace text} -- \texttt{\textbackslash texttt\{...\}}
  \item \textsc{Small Caps} -- \texttt{\textbackslash textsc\{...\}}
  \item \sout{Strikethrough text} -- \texttt{\textbackslash sout\{...\}} (requires \texttt{ulem} package)
\end{itemize}

\subsection{Quotations}
\label{sec:HowTo_Quotations}

For short inline quotes, use quotation marks: ``This is a quote.''

For longer block quotes, use the \texttt{quote} or \texttt{quotation} environment:

\begin{quote}
  \colorbox{gray!15}{\parbox{\dimexpr\linewidth-2\fboxsep}{%
    This is a block quotation. It is indented from both margins and is useful for longer quoted passages from other sources.%
  }}
\end{quote}

%=============================================================================
\section{Comments and TODOs}
\label{sec:HowTo_Comments}
%=============================================================================

\subsection{LaTeX Comments}
\label{sec:HowTo_LaTeXComments}

Use \texttt{\%} for single-line comments. For multi-line comments, use the \texttt{comment} environment:

% This is a single-line comment
\begin{comment}
This entire block is commented out.
It will not appear in the output.
Useful for temporarily removing large sections.
\end{comment}

\subsection{TODO Notes}
\label{sec:HowTo_TodoNotes}

This template provides inline TODO notes for review:

\mycomment{This is an example TODO note that appears inline in the document.}

These notes are visible during writing but can be disabled for final submission.

%=============================================================================
\section{Hyperlinks and URLs}
\label{sec:HowTo_Hyperlinks}
%=============================================================================

The \texttt{hyperref} package enables clickable links:

\begin{itemize}
  \item URLs: \url{https://www.latex-project.org/}
  \item Named links: \href{https://www.overleaf.com}{Overleaf Online Editor}
  \item Email: \href{mailto:example@university.edu}{example@university.edu}
\end{itemize}

All cross-references, citations, and table of contents entries are automatically hyperlinked.

%=============================================================================
\section{Best Practices}
\label{sec:HowTo_BestPractices}
%=============================================================================

\begin{enumerate}
  \item \textbf{Use meaningful labels}: Prefix labels with the type (e.g., \texttt{fig:}, \texttt{tab:}, \texttt{eq:}, \texttt{sec:}, \texttt{lst:}, \texttt{alg:}).

  \item \textbf{Compile multiple times}: Run the compiler at least twice to resolve cross-references. For bibliography, run: \texttt{pdflatex} $\rightarrow$ \texttt{biber} $\rightarrow$ \texttt{pdflatex} $\rightarrow$ \texttt{pdflatex}.

  \item \textbf{Use non-breaking spaces}: Use \texttt{\~{}} before references to prevent line breaks: \texttt{Figure\~{}\textbackslash ref\{fig:example\}}.

  \item \textbf{Keep source files organized}: Use separate \texttt{.tex} files for each chapter/section.

  \item \textbf{Use vector graphics}: Prefer PDF or SVG for diagrams and figures when possible.

  \item \textbf{Backup your work}: Use version control (Git) and regular backups.

  \item \textbf{Check for overfull boxes}: Review warnings about overfull \texttt{hbox} or \texttt{vbox} and adjust text/figures accordingly.
\end{enumerate}
      %%%%%% <<<<<< nach dem ersten Ansehen, bitte diese Zeile auskommentieren
\chapter{Organization Profile and Internship Setup}
\label{ch:OrgProfile}

% Page budget: 4 pages (Pages 4-7)
% Covers guideline: Profile of organization, Tasks assigned by senior role, Intra-organization communication (explicit)

%=============================================================================
\section{Organization Profile}
\label{sec:OrgProfile_Profile}
%=============================================================================

% 1 page
% Domain, teams, product area, constraints



%=============================================================================
\section{Role, Responsibilities, and Stakeholders}
\label{sec:OrgProfile_Role}
%=============================================================================

% 1 page
% Senior roles, review loops, what you owned

% Figure placeholder: F3 - Stakeholder / communication diagram (you ↔ teams)
% \begin{figure}[htbp]
%   \centering
%   % \includegraphics[width=0.8\textwidth]{Figures/stakeholder-diagram.pdf}
%   \caption{Stakeholder and communication diagram showing the intern's interactions with various teams.}
%   \label{fig:stakeholder_diagram}
% \end{figure}


%=============================================================================
\section{Communication and Execution Model}
\label{sec:OrgProfile_Communication}
%=============================================================================

% 1 page
% How you interacted: requirement intake, reviews, demos, iteration cycles
% What "handoff" and "acceptance" meant



%=============================================================================
\section{High-Level Challenges and Constraints}
\label{sec:OrgProfile_Challenges}
%=============================================================================

% 1 page
% Resource limits, time constraints, operational constraints

% Table placeholder: T2 - Constraints and how you handled them
% \begin{table}[htbp]
%   \centering
%   \caption{Key constraints encountered and mitigation strategies.}
%   \label{tab:Constraints}
%   \begin{tabular}{lp{5cm}p{5cm}}
%     \toprule
%     \textbf{Constraint} & \textbf{Impact} & \textbf{Mitigation} \\
%     \midrule
%     Time & ... & ... \\
%     Resources & ... & ... \\
%     Operational & ... & ... \\
%     \bottomrule
%   \end{tabular}
% \end{table}
\input{TeXFiles/030_Hintergrund}
\chapter{Track B Overview: Agent Platform Goals and Requirements}
\label{ch:TrackBOverview}

% Page budget: 3 pages (Pages 27-29)
% This is the "industrial core" - the LangGraph Agent Platform

%=============================================================================
\section{Business Capability and User Journeys}
\label{sec:TrackB_BusinessCapability}
%=============================================================================

% 1.5 pages
% Target users: QA, PM/PE, Ops
% Primary journeys:
% - Requirement → Test Cases (HITL)
% - Test Case → Test Steps (HITL)

% Figure placeholder: F15 - User journey swimlane diagram (HITL)
% \begin{figure}[htbp]
%   \centering
%   % \includegraphics[width=\textwidth]{Figures/user-journey-swimlane.pdf}
%   \caption{User journey swimlane diagram showing human-in-the-loop gates at each decision point.}
%   \label{fig:user_journey_swimlane}
% \end{figure}


%=============================================================================
\section{Requirements Specification}
\label{sec:TrackB_Requirements}
%=============================================================================

% 1.5 pages

\subsection{Functional Requirements}
\label{sec:TrackB_FunctionalReqs}

% Iterative loops, edits, regeneration


\subsection{Non-Functional Requirements}
\label{sec:TrackB_NFRs}

% Multi-tenant governance, observability best-effort, CI/CD, reliability


% Figure placeholder: F16 - Requirements-to-mechanisms map
% \begin{figure}[htbp]
%   \centering
%   % \includegraphics[width=\textwidth]{Figures/requirements-mechanisms-map.pdf}
%   \caption{Mapping of requirements to implementation mechanisms in the agent platform.}
%   \label{fig:requirements_mechanisms}
% \end{figure}

% Table placeholder: T8 - Journey → Graph → Interrupt points → Outputs
% \begin{table}[htbp]
%   \centering
%   \caption{Mapping user journeys to graphs, interrupt points, and outputs.}
%   \label{tab:JourneyGraphMapping}
%   \begin{tabular}{llll}
%     \toprule
%     \textbf{Journey} & \textbf{Graph} & \textbf{Interrupts} & \textbf{Outputs} \\
%     \midrule
%     Requirement → Cases & Graph A & Review gate & Test cases \\
%     Case → Steps & Graph B & Edit gate & Test steps \\
%     \bottomrule
%   \end{tabular}
% \end{table}
% =============================================================================
% TRACK B: LangGraph Agent Platform Engineering
% Page budget: 22 pages (Pages 27-48) - but split across multiple files
% This file covers: Industrial Architecture (7 pages) + Graph Design (7 pages) = 14 pages
% =============================================================================

\chapter{LangGraph Industrial Architecture}
\label{ch:LangGraphArchitecture}

% Page budget: 7 pages (Pages 30-36)
% This is the centerpiece for professionalism

%=============================================================================
\section{Component Architecture}
\label{sec:ComponentArchitecture}
%=============================================================================

% 2 pages
% - LangGraph API service
% - Redis governance
% - Postgres checkpoints
% - LLM provider
% - Langfuse observability
% - Clients/consumers

% Figure placeholder: F17 - Component diagram (LangGraph platform)
% \begin{figure}[htbp]
%   \centering
%   % \includegraphics[width=\textwidth]{Figures/langgraph-component-diagram.pdf}
%   \caption{Component diagram showing the LangGraph platform architecture with all major services.}
%   \label{fig:langgraph_component}
% \end{figure}


%=============================================================================
\section{Deployment Architecture (Cloud Run)}
\label{sec:DeploymentArchitecture}
%=============================================================================

% 2 pages
% Service boundaries: API service + cron service
% Scaling model, concurrency, cold starts considerations
% Network/secrets policy

% Figure placeholder: F18 - Deployment topology (Cloud Run + dependencies)
% \begin{figure}[htbp]
%   \centering
%   % \includegraphics[width=\textwidth]{Figures/deployment-topology-cloudrun.pdf}
%   \caption{Deployment topology on GCP Cloud Run showing service boundaries and dependencies.}
%   \label{fig:deployment_topology}
% \end{figure}


%=============================================================================
\section{Runtime Request Lifecycle}
\label{sec:RuntimeLifecycle}
%=============================================================================

% 2 pages
% Invoke, interrupts, resume, checkpoints, callbacks

% Figure placeholder: F19 - Sequence diagram (invoke/resume lifecycle)
% \begin{figure}[htbp]
%   \centering
%   % \includegraphics[width=\textwidth]{Figures/invoke-resume-sequence.pdf}
%   \caption{Sequence diagram showing the invoke/resume lifecycle with interrupt and checkpoint handling.}
%   \label{fig:invoke_resume_sequence}
% \end{figure}

% Figure placeholder: F20 - "Control points" diagram (where governance/validation happens)
% \begin{figure}[htbp]
%   \centering
%   % \includegraphics[width=0.9\textwidth]{Figures/control-points.pdf}
%   \caption{Control points diagram showing where governance and validation occur in the request flow.}
%   \label{fig:control_points}
% \end{figure}


%=============================================================================
\section{Key Architectural Decisions}
\label{sec:KeyArchDecisions}
%=============================================================================

% 1 page, ADR summary
% State machines, HITL interrupts, structured outputs, best-effort tracing, pooling

% Table placeholder: T9 - ADR summary table (decision → options → trade-offs)
% \begin{table}[htbp]
%   \centering
%   \caption{Summary of key architectural decisions.}
%   \label{tab:ADRSummary}
%   \begin{tabular}{lp{4cm}p{5cm}}
%     \toprule
%     \textbf{Decision} & \textbf{Options Considered} & \textbf{Trade-offs} \\
%     \midrule
%     State machines & Custom, LangGraph & LangGraph: built-in persistence \\
%     HITL interrupts & Polling, Interrupts & Interrupts: cleaner UX \\
%     Structured outputs & Prompt, Schema & Schema: validation + reliability \\
%     Tracing & Strict, Best-effort & Best-effort: no blocking \\
%     \bottomrule
%   \end{tabular}
% \end{table}


%=============================================================================
%=============================================================================
\chapter{Graph Design and Runtime Behavior}
\label{ch:GraphDesign}

% Page budget: 7 pages (Pages 37-43)
% This section "differentiates agent from workflow" by showing state machines and formal contracts

%=============================================================================
\section{State and Schema Design}
\label{sec:StateSchemaDesign}
%=============================================================================

% 2 pages
% State schema, reducers
% Structured output contracts (with_structured_output)
% Validation strategy and failure recovery

% Figure placeholder: F21 - State schema + reducers diagram (state ownership)
% \begin{figure}[htbp]
%   \centering
%   % \includegraphics[width=0.9\textwidth]{Figures/state-schema-reducers.pdf}
%   \caption{State schema and reducers diagram showing state ownership and update patterns.}
%   \label{fig:state_schema}
% \end{figure}

% Figure placeholder: F22 - Structured output boundary diagram (schema contracts)
% \begin{figure}[htbp]
%   \centering
%   % \includegraphics[width=0.8\textwidth]{Figures/structured-output-boundary.pdf}
%   \caption{Structured output contract boundaries showing schema validation points.}
%   \label{fig:structured_output}
% \end{figure}


%=============================================================================
\section{Graph A: Requirement to Test Cases}
\label{sec:GraphA}
%=============================================================================

% 2 pages
% Transitions, regen modes, HITL gate, loop

% Figure placeholder: F23 - State machine diagram: Test Cases graph
% \begin{figure}[htbp]
%   \centering
%   % \includegraphics[width=\textwidth]{Figures/graph-a-state-machine.pdf}
%   \caption{State machine diagram for Graph A: Requirement → Test Cases.}
%   \label{fig:graph_a_state_machine}
% \end{figure}

% Table placeholder: T10 - Action → transition → node → state delta (Graph A)
% \begin{table}[htbp]
%   \centering
%   \caption{Graph A state transitions and their effects.}
%   \label{tab:GraphATransitions}
%   \begin{tabular}{llll}
%     \toprule
%     \textbf{Action} & \textbf{Transition} & \textbf{Node} & \textbf{State Delta} \\
%     \midrule
%     Generate & START → generate & generate\_cases & cases created \\
%     Review & generate → review & review\_gate & awaiting input \\
%     Approve & review → END & finalize & output ready \\
%     Reject & review → generate & regenerate & cases cleared \\
%     \bottomrule
%   \end{tabular}
% \end{table}


%=============================================================================
\section{Graph B: Test Case to Test Steps}
\label{sec:GraphB}
%=============================================================================

% 2 pages
% Edit semantics, reject + feedback + regen loop

% Figure placeholder: F24 - State machine diagram: Test Steps graph
% \begin{figure}[htbp]
%   \centering
%   % \includegraphics[width=\textwidth]{Figures/graph-b-state-machine.pdf}
%   \caption{State machine diagram for Graph B: Test Case → Test Steps.}
%   \label{fig:graph_b_state_machine}
% \end{figure}

% Table placeholder: T11 - Edit commands → validation → effect (Graph B)
% \begin{table}[htbp]
%   \centering
%   \caption{Edit command semantics and validation for Graph B.}
%   \label{tab:GraphBEditCommands}
%   \begin{tabular}{llp{5cm}}
%     \toprule
%     \textbf{Command} & \textbf{Validation} & \textbf{Effect} \\
%     \midrule
%     ADD & Step schema valid & Insert at position \\
%     DELETE & Step ID exists & Remove from list \\
%     UPDATE & Step ID exists, schema valid & Replace in place \\
%     \bottomrule
%   \end{tabular}
% \end{table}

% Table placeholder: T12 - Failure modes and recovery (schema/LLM/provider)
% \begin{table}[htbp]
%   \centering
%   \caption{Failure modes and recovery strategies.}
%   \label{tab:FailureRecovery}
%   \begin{tabular}{llp{5cm}}
%     \toprule
%     \textbf{Failure Type} & \textbf{Source} & \textbf{Recovery} \\
%     \midrule
%     Schema validation & LLM output & Retry with stricter prompt \\
%     Provider timeout & API & Retry with backoff \\
%     State corruption & Bug & Rollback to checkpoint \\
%     \bottomrule
%   \end{tabular}
% \end{table}


%=============================================================================
\section{HITL Interrupt and Resume Lifecycle}
\label{sec:HITLLifecycle}
%=============================================================================

% 1 page
% What is stored, what is validated, thread lifecycle

% Figure placeholder: F25 - Interrupt/resume timeline + checkpoint diagram
% \begin{figure}[htbp]
%   \centering
%   % \includegraphics[width=\textwidth]{Figures/hitl-lifecycle-timeline.pdf}
%   \caption{Timeline showing the HITL interrupt/resume lifecycle with checkpoint storage.}
%   \label{fig:hitl_timeline}
% \end{figure}

% =============================================================================
% Governance & Observability + CI/CD & Operations
% Page budget: 5 pages (Pages 44-48)
% =============================================================================

\chapter{Governance and Observability}
\label{ch:Governance}

% Page budget: 3 pages (Pages 44-46)

%=============================================================================
\section{Token Usage Tracking}
\label{sec:TokenTracking}
%=============================================================================

% Callback → usage_metadata → Redis counters

% Figure placeholder: F26 - Token governance flow
% \begin{figure}[htbp]
%   \centering
%   % \includegraphics[width=\textwidth]{Figures/token-governance-flow.pdf}
%   \caption{Token governance flow showing callback processing, Redis counter updates, and enforcement points.}
%   \label{fig:token_governance}
% \end{figure}


%=============================================================================
\section{Limit Enforcement}
\label{sec:LimitEnforcement}
%=============================================================================

% Pre-call checks, block behavior, multi-tenant safety


%=============================================================================
\section{Observability with Langfuse}
\label{sec:ObservabilityLangfuse}
%=============================================================================

% Best-effort observability approach
% Attach handler only if auth passes
% Traces enriched with org/user/project
% Minimum viable monitoring signals

% Figure placeholder: F27 - Observability integration diagram (traces + metrics + tags)
% \begin{figure}[htbp]
%   \centering
%   % \includegraphics[width=\textwidth]{Figures/observability-architecture.pdf}
%   \caption{Observability integration showing trace collection, metrics aggregation, and tenant-aware tagging.}
%   \label{fig:observability_architecture}
% \end{figure}

% Table placeholder: T13 - Signals table (metric → source → action)
% \begin{table}[htbp]
%   \centering
%   \caption{Monitoring signals and operator actions.}
%   \label{tab:MonitoringSignals}
%   \begin{tabular}{lll}
%     \toprule
%     \textbf{Metric} & \textbf{Source} & \textbf{Operator Action} \\
%     \midrule
%     ... & ... & ... \\
%     \bottomrule
%   \end{tabular}
% \end{table}

% ADR Box: observability best-effort principle
% ADR Box: token governance design (Redis key schema)


%=============================================================================
%=============================================================================
\chapter{CI/CD and Operations}
\label{ch:CICD}

% Page budget: 2 pages (Pages 47-48)

%=============================================================================
\section{CI/CD Pipeline Overview}
\label{sec:PipelineOverview}
%=============================================================================

% Build → test → containerize → deploy to Cloud Run
% Versioning and rollback strategy

% Figure placeholder: F28 - CI/CD pipeline diagram
% \begin{figure}[htbp]
%   \centering
%   % \includegraphics[width=\textwidth]{Figures/cicd-pipeline.pdf}
%   \caption{CI/CD pipeline showing build, test, containerization, and Cloud Run deployment stages.}
%   \label{fig:cicd_pipeline}
% \end{figure}


%=============================================================================
\section{Operational Runbook and Hardening Checklist}
\label{sec:OperationalRunbook}
%=============================================================================

% Common failure modes
% Monthly maintenance
% Security checklist: secrets, network, auth, validation

% Figure placeholder: F29 - Hardening checklist as control-map diagram
% \begin{figure}[htbp]
%   \centering
%   % \includegraphics[width=0.9\textwidth]{Figures/hardening-checklist.pdf}
%   \caption{Hardening control map showing security controls, network policies, and validation layers.}
%   \label{fig:hardening_checklist}
% \end{figure}

% Table placeholder: T14 - Runbook quick reference (symptom → likely cause → action)
% \begin{table}[htbp]
%   \centering
%   \caption{Operational runbook quick reference.}
%   \label{tab:RunbookReference}
%   \begin{tabular}{lll}
%     \toprule
%     \textbf{Symptom} & \textbf{Likely Cause} & \textbf{Action} \\
%     \midrule
%     ... & ... & ... \\
%     \bottomrule
%   \end{tabular}
% \end{table}

% ADR Box: Cloud Run deployment choice






\chapter{Results, Skills, Lessons, and Personal Commentary}
\label{ch:Results}

% Page budget: 2 pages (Pages 49-50)
% This must satisfy the guideline items explicitly:
% - Results produced (deliverables list)
% - Skills developed (engineering + collaboration)
% - Key learning points and decision-making principles
% - Personal comments

%=============================================================================
\section{Results Produced}
\label{sec:ResultsProduced}
%=============================================================================

% Page budget: 0.5 page
% Track A deliverables
% Track B deliverables

% Table placeholder: T15 - Results matrix (Task/Track → deliverable → impact)
% \begin{table}[htbp]
%   \centering
%   \caption{Results matrix showing deliverables and their impact.}
%   \label{tab:ResultsMatrix}
%   \begin{tabular}{llll}
%     \toprule
%     \textbf{Track/Task} & \textbf{Deliverable} & \textbf{Impact} & \textbf{Status} \\
%     \midrule
%     \multicolumn{4}{l}{\textit{Track A: Workflow Automation \& Infra R\&D}} \\
%     ERP Integration & n8n Workflows & ... & Complete \\
%     Model Serving & Serving Evaluation Report & ... & Complete \\
%     MCP Exploration & Interoperability Research & ... & Complete \\
%     \midrule
%     \multicolumn{4}{l}{\textit{Track B: LangGraph Agent Platform}} \\
%     Agent Platform & LangGraph Server & ... & Complete \\
%     Governance & Token Tracking System & ... & Complete \\
%     CI/CD & Deployment Pipeline & ... & Complete \\
%     \bottomrule
%   \end{tabular}
% \end{table}


%=============================================================================
\section{Skills Developed}
\label{sec:SkillsDeveloped}
%=============================================================================

% Page budget: 0.5 page
% Engineering: architecture, CI/CD, observability, governance
% Collaboration and review loop

% Table placeholder: T15 - Skills gained (technical + professional)
% \begin{table}[htbp]
%   \centering
%   \caption{Technical and professional skills developed during the internship.}
%   \label{tab:SkillsGained}
%   \begin{tabular}{lp{8cm}}
%     \toprule
%     \textbf{Category} & \textbf{Skills} \\
%     \midrule
%     Architecture & System design, component boundaries, deployment topology \\
%     CI/CD & Pipeline design, containerization, Cloud Run deployment \\
%     Observability & Trace integration, metrics design, monitoring signals \\
%     Governance & Token tracking, limit enforcement, multi-tenant safety \\
%     Collaboration & Stakeholder communication, review loops, documentation \\
%     \bottomrule
%   \end{tabular}
% \end{table}


%=============================================================================
\section{Lessons Learned and Personal Commentary}
\label{sec:LessonsLearned}
%=============================================================================

% Page budget: 1 page
% 8-12 compact lessons, each tied to a decision point
% "Principles I will keep" (production mindset)

% Table placeholder: T16 - Lessons learned (lesson → where it applies)
% \begin{table}[htbp]
%   \centering
%   \caption{Key lessons learned and their application.}
%   \label{tab:LessonsLearned}
%   \begin{tabular}{lp{7cm}}
%     \toprule
%     \textbf{Lesson} & \textbf{Application} \\
%     \midrule
%     Evidence-first decisions & Always benchmark before committing to a solution \\
%     Workflow vs Agent distinction & Choose the right paradigm for the problem \\
%     Best-effort observability & Production systems need visibility, even imperfect \\
%     Schema contracts & Structured outputs reduce integration failures \\
%     ... & ... \\
%     \bottomrule
%   \end{tabular}
% \end{table}


\input{TeXFiles/080_Zusammenfassung}

\begin{appendix}
  \chapter{Appendix}
\label{ch:Appendix}

% Appendices are optional and outside the 50-page main content

%=============================================================================
\section{Code Samples}
\label{sec:Appendix_Code}
%=============================================================================

% Key code snippets that support the report


%=============================================================================
\section{Additional Diagrams}
\label{sec:Appendix_Diagrams}
%=============================================================================

% Detailed diagrams that don't fit in the main text


%=============================================================================
\section{Configuration Files}
\label{sec:Appendix_Config}
%=============================================================================

% langgraph.json, Dockerfile excerpts, CI/CD configs


%=============================================================================
\section{Runbook Excerpts}
\label{sec:Appendix_Runbook}
%=============================================================================

% Operational procedures for common tasks


\end{appendix}

\clearpage

% Only print glossary entries that are referenced in the text
% To print all entries (even unreferenced), uncomment the line below:
% \glsaddall

\ifdefined\ThesisLanguageIsEnglish
  \printglossary[type=\acronymtype, title=List of Abbreviations, toctitle=List of Abbreviations]
\else
  \printglossary[type=\acronymtype, title=Abkürzungsverzeichnis, toctitle=Abkürzungsverzeichnis]
\fi
%\printglossary[type=acronym,title=Abbreviations]
%\addcontentsline{toc}{section}{Abbreviations} 
%\addcontentsline{toc}{chapter*}{Verzeichnis: Akronyme} 
\clearpage

\ifdefined\ThesisLanguageIsEnglish
  \printglossary[title=Glossary, toctitle=Glossary]
\else
  % Ensure all glossary entries are included even if not referenced in text
  \printglossary[title=Glossar, toctitle=Glossar]
\fi

%Falls man die Überschrift des Literaturverzeichnisses
%aendern moechte, geht das durch Verwendung der folgenden Zeile:
%\renewcommand*{\refname}{Literaturverzeichnis}

%\bibliographystyle{plain} % Nummern in eckigen Klammern
%%\bibliographystyle{alpha} % Anfangsbuchstaben Erstautor und Jahr

% Wenn man folgenden Stil verwenden will, dann muss
% \usepackage{harvard} vor \begin{document} aktiviert werden:
%\bibliographystyle{agsm} % (Autoren in runden Klammern)
%\newpage
\clearpage
%\phantomsection % Da sonst falsche Verlinkung im Inhaltsverzeichnis
\printbibliography[heading=bibintoc]    % Ohne Kapitelnummer /-buchstabe
%\printbibliography[heading=bibnumbered]    % Mit Kapitelnummer /-buchstabe
\end{document}
