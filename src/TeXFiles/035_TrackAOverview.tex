%=============================================================================
\chapter{Track A Overview: Workflow Automation and Infrastructure R\&D}
\label{ch:TrackAOverview}
%=============================================================================
% Page budget: 2 pages (Pages 11--12)
% Track A is presented as an independent engineering track focused on
% workflow automation, infrastructure experimentation, and interoperability
% research. No architectural or runtime dependency on LangGraph is assumed.
%=============================================================================

This chapter provides an overview of Track A, which encompasses workflow
automation and infrastructure-oriented research and development conducted
during the internship. Track A is treated as an independent engineering
initiative with its own objectives, scope, and architectural considerations.
The work in this track focused on understanding how AI-enabled capabilities
can be integrated into existing systems through deterministic workflows,
as well as evaluating infrastructure patterns relevant to serving and
operating such systems in practice.

%=============================================================================
\section{Track A Goals and Scope}
\label{sec:TrackA_Goals}
%=============================================================================

The primary goal of Track A was to explore and validate practical approaches
for integrating AI-assisted functionality into existing enterprise systems
and workflows, while maintaining operational safety, traceability, and
predictability.

Specifically, Track A addressed the following problem areas:

\begin{itemize}
    \item \textbf{Workflow automation for enterprise systems}, with a focus
          on integrating AI-assisted steps into existing ERP-related
          processes using deterministic orchestration.
    \item \textbf{Infrastructure experimentation for model serving}, including
          local deployment patterns and comparative evaluation of serving
          approaches from an operational perspective.
    \item \textbf{Tooling interoperability research}, explored through
          experimentation with the Model Context Protocol (MCP) as a
          standardized interface for exposing tools, resources, and prompts
          to AI systems.
\end{itemize}

The intent of this track was exploratory and evaluative rather than
product-defining. The work emphasized understanding trade-offs, operational
constraints, and failure modes associated with each approach, rather than
building a single end-to-end production system.

\subsection{Explicit Scope}
\label{subsec:TrackA_ExplicitScope}

Track A includes the following concrete activities:
\begin{itemize}
    \item Design and implementation of AI-enabled workflow automation using
          \texttt{n8n}, integrated with existing ERP processes.
    \item Evaluation of local model serving and enterprise-oriented deployment
          options, including performance and operational considerations.
    \item Experimental investigation of MCP as a mechanism for tool and
          context interoperability across AI systems.
\end{itemize}

\subsection{Out-of-Scope}
\label{subsec:TrackA_OutOfScope}

To maintain clarity and architectural separation, the following items are
explicitly out of scope for Track A:
\begin{itemize}
    \item Agent orchestration, stateful reasoning, or human-in-the-loop control
          implemented via LangGraph.
    \item Production-grade agent lifecycle management, governance, or
          observability mechanisms introduced in Track B.
    \item Tight coupling between workflow automation components and agent-based
          systems.
\end{itemize}

This separation ensures that conclusions drawn in Track A remain applicable
to workflow and infrastructure design in general, without assuming an
agent-based execution model.

%=============================================================================
\section{Track A Architecture Overview}
\label{sec:TrackA_Architecture}
%=============================================================================

Track A employed a modular and loosely coupled architecture that allowed
individual components to be evaluated in isolation. The architecture was
designed to support experimentation while minimizing the risk of unintended
side effects on existing systems.

At a high level, the architecture consists of four conceptual layers:

\begin{enumerate}
    \item \textbf{Enterprise system layer}, representing existing ERP systems
          and business processes that act as sources of events or consumers
          of automated actions.
    \item \textbf{Workflow orchestration layer}, implemented using \texttt{n8n},
          responsible for deterministic execution of automation logic,
          conditional branching, retries, and error handling.
    \item \textbf{Infrastructure and serving sandbox}, used to evaluate local
          and enterprise-oriented model serving patterns under controlled
          conditions.
    \item \textbf{Interoperability research sandbox}, where MCP was explored as
          a standardized interface for exposing tools and contextual resources
          to AI-enabled components.
\end{enumerate}

These layers were intentionally kept independent to allow focused evaluation
of each concern. In particular, workflow orchestration was treated as a
deterministic control plane, while infrastructure and interoperability
experiments were conducted in isolation from production workflows.

\begin{figure}[htbp]
    \centering
    % \includegraphics[width=\textwidth]{Figures/track-a-context.pdf}
    \caption{Track A system context illustrating workflow automation components,
        infrastructure experimentation, and interoperability research sandbox.}
    \label{fig:track_a_context}
\end{figure}

This architectural separation enabled:
\begin{itemize}
    \item clear reasoning about failure modes and recovery paths in automated
          workflows,
    \item controlled evaluation of serving and deployment options without
          production risk,
    \item and exploratory investigation of interoperability mechanisms without
          constraining later architectural decisions.
\end{itemize}

The findings and lessons from Track A informed general engineering judgment and
operational awareness, but did not impose architectural dependencies on the
agent-based systems developed in Track B.