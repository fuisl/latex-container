\chapter*{\LaTeX{} How-To Guide}
\label{ch:LaTeXHowTo}
\addcontentsline{toc}{chapter}{\LaTeX{} How-To Guide}

This chapter provides a comprehensive guide on how to use \LaTeX{} for writing technical reports. It covers essential features including document structure, cross-referencing, figures, tables, mathematical equations, code listings, citations, and more.

%=============================================================================
\section{Document Structure and Cross-References}
\label{sec:HowTo_Structure}
%=============================================================================

\subsection{Chapters, Sections, and Subsections}
\label{sec:HowTo_Sections}

\LaTeX{} provides a hierarchical structure for organizing your document:

\begin{itemize}
  \item \texttt{\textbackslash chapter\{Title\}} -- Main divisions (in report/book classes)
  \item \texttt{\textbackslash section\{Title\}} -- Major sections within chapters
  \item \texttt{\textbackslash subsection\{Title\}} -- Subsections
  \item \texttt{\textbackslash subsubsection\{Title\}} -- Further subdivisions
  \item \texttt{\textbackslash paragraph\{Title\}} -- Named paragraphs
\end{itemize}

\subsection{Labels and Cross-References}
\label{sec:HowTo_CrossRef}

Every element that you want to reference should have a \texttt{\textbackslash label\{key\}} command. You can then reference it using:

\begin{itemize}
  \item \texttt{\textbackslash ref\{key\}} -- Returns the number (e.g., ``Section~\ref{sec:HowTo_CrossRef}'')
  \item \texttt{\textbackslash pageref\{key\}} -- Returns the page number
  \item \texttt{\textbackslash autoref\{key\}} -- Automatically includes the type (e.g., ``Section'')
\end{itemize}

For example, we are currently in Section~\ref{sec:HowTo_Structure}, specifically in Subsection~\ref{sec:HowTo_CrossRef}.

%=============================================================================
\section{Figures and Graphics}
\label{sec:HowTo_Figures}
%=============================================================================

\subsection{Basic Figure Insertion}
\label{sec:HowTo_BasicFigure}

Figures are inserted using the \texttt{figure} environment with the \texttt{graphicx} package:

\begin{figure}[htbp]
  \centering
  \includegraphics[width=0.5\textwidth]{Figures/vgu-logo.pdf}
  \caption{Example of a centered figure with caption.}
  \label{fig:ExampleLogo}
\end{figure}

The placement specifiers \texttt{[htbp]} control where \LaTeX{} tries to place the figure:
\begin{itemize}
  \item \texttt{h} -- Here (approximately where it appears in the source)
  \item \texttt{t} -- Top of the page
  \item \texttt{b} -- Bottom of the page
  \item \texttt{p} -- On a separate page of floats
  \item \texttt{!} -- Override internal parameters
\end{itemize}

To reference Figure~\ref{fig:ExampleLogo}, use \texttt{\textbackslash ref\{fig:ExampleLogo\}}.

\subsection{Subfigures}
\label{sec:HowTo_Subfigures}

For multiple related figures, use the \texttt{subcaption} package:

\begin{figure}[htbp]
  \centering
  \begin{subfigure}[b]{0.45\textwidth}
    \centering
    \includegraphics[width=0.8\textwidth]{Figures/vgu-logo.pdf}
    \caption{First subfigure.}
    \label{fig:sub1}
  \end{subfigure}
  \hfill
  \begin{subfigure}[b]{0.45\textwidth}
    \centering
    \includegraphics[width=0.8\textwidth]{Figures/company-logo.png}
    \caption{Second subfigure.}
    \label{fig:sub2}
  \end{subfigure}
  \caption{Example of subfigures: (a) VGU Logo, (b) Company Logo.}
  \label{fig:Subfigures}
\end{figure}

Reference individual subfigures as Figure~\ref{fig:sub1} or the whole figure as Figure~\ref{fig:Subfigures}.

%=============================================================================
\section{Tables}
\label{sec:HowTo_Tables}
%=============================================================================

\subsection{Basic Tables with Booktabs}
\label{sec:HowTo_BasicTables}

Professional tables use the \texttt{booktabs} package for clean horizontal rules:

\begin{table}[htbp]
  \centering
  \caption{Example of a professional table using booktabs.}
  \label{tab:BasicTable}
  \begin{tabular}{lcc}
    \toprule
    \textbf{Method}  & \textbf{Accuracy (\%)} & \textbf{Time (s)} \\
    \midrule
    Baseline         & 85.2                   & 1.23              \\
    Proposed         & 92.7                   & 0.89              \\
    State-of-the-art & 91.3                   & 2.15              \\
    \bottomrule
  \end{tabular}
\end{table}

Key commands from \texttt{booktabs}:
\begin{itemize}
  \item \texttt{\textbackslash toprule} -- Top rule (thicker)
  \item \texttt{\textbackslash midrule} -- Middle rule (thinner)
  \item \texttt{\textbackslash bottomrule} -- Bottom rule (thicker)
  \item \texttt{\textbackslash cmidrule\{i-j\}} -- Partial rule from column $i$ to $j$
\end{itemize}

\subsection{Multi-row and Multi-column Cells}
\label{sec:HowTo_MultiRowCol}

Use \texttt{multirow} and \texttt{multicolumn} for spanning cells:

\begin{table}[htbp]
  \centering
  \caption{Table with multi-row and multi-column cells.}
  \label{tab:MultiTable}
  \begin{tabular}{lccc}
    \toprule
                         & \multicolumn{3}{c}{\textbf{Metrics}}                                       \\
    \cmidrule(lr){2-4}
    \textbf{Model}       & \textbf{Precision}                   & \textbf{Recall} & \textbf{F1-Score} \\
    \midrule
    \multirow{2}{*}{CNN} & 0.89                                 & 0.87            & 0.88              \\
                         & 0.91                                 & 0.85            & 0.88              \\
    \midrule
    \multirow{2}{*}{RNN} & 0.85                                 & 0.90            & 0.87              \\
                         & 0.86                                 & 0.88            & 0.87              \\
    \bottomrule
  \end{tabular}
\end{table}

%=============================================================================
\section{Mathematical Equations}
\label{sec:HowTo_Math}
%=============================================================================

\subsection{Inline and Display Math}
\label{sec:HowTo_InlineMath}

Inline math is written between dollar signs: The equation $E = mc^2$ is famous. For display equations, use the \texttt{equation} environment:

\begin{equation}
  f(x) = \int_{-\infty}^{\infty} \hat{f}(\xi) e^{2\pi i \xi x} \, d\xi
  \label{eq:FourierTransform}
\end{equation}

Equation~\ref{eq:FourierTransform} shows the inverse Fourier transform.

\subsection{Multi-line Equations with Alignment}
\label{sec:HowTo_AlignedMath}

Use the \texttt{align} environment for aligned multi-line equations:

\begin{align}
  \nabla \cdot \mathbf{E}  & = \frac{\rho}{\varepsilon_0} \label{eq:Maxwell1}                                                    \\
  \nabla \cdot \mathbf{B}  & = 0 \label{eq:Maxwell2}                                                                             \\
  \nabla \times \mathbf{E} & = -\frac{\partial \mathbf{B}}{\partial t} \label{eq:Maxwell3}                                       \\
  \nabla \times \mathbf{B} & = \mu_0 \mathbf{J} + \mu_0 \varepsilon_0 \frac{\partial \mathbf{E}}{\partial t} \label{eq:Maxwell4}
\end{align}

These are Maxwell's equations (Equations~\ref{eq:Maxwell1}--\ref{eq:Maxwell4}).

\subsection{Matrices and Arrays}
\label{sec:HowTo_Matrices}

Various matrix environments are available:

\begin{equation}
  \mathbf{A} = \begin{bmatrix}
    a_{11} & a_{12} & a_{13} \\
    a_{21} & a_{22} & a_{23} \\
    a_{31} & a_{32} & a_{33}
  \end{bmatrix}
  \qquad
  \mathbf{B} = \begin{pmatrix}
    b_1 \\
    b_2 \\
    b_3
  \end{pmatrix}
  \label{eq:Matrices}
\end{equation}

Other matrix styles: \texttt{matrix} (no delimiters), \texttt{vmatrix} (vertical bars for determinants), \texttt{Vmatrix} (double vertical bars).

\subsection{Common Mathematical Symbols}
\label{sec:HowTo_MathSymbols}

\begin{itemize}
  \item Greek letters: $\alpha, \beta, \gamma, \delta, \epsilon, \theta, \lambda, \mu, \sigma, \phi, \omega$
  \item Operators: $\sum_{i=1}^{n}$, $\prod_{j=1}^{m}$, $\int_a^b$, $\lim_{x \to \infty}$
  \item Relations: $\leq, \geq, \neq, \approx, \equiv, \propto, \sim$
  \item Sets: $\in, \notin, \subset, \subseteq, \cup, \cap, \emptyset, \mathbb{R}, \mathbb{N}$
  \item Arrows: $\rightarrow, \leftarrow, \Rightarrow, \Leftrightarrow, \mapsto$
  \item Fractions: $\frac{a}{b}$, $\dfrac{dy}{dx}$
  \item Roots: $\sqrt{x}$, $\sqrt[n]{x}$
\end{itemize}

%=============================================================================
\section{Code Listings}
\label{sec:HowTo_Code}
%=============================================================================

\subsection{Basic Code Listing}
\label{sec:HowTo_BasicCode}

Use the \texttt{listings} package for syntax-highlighted code:

\begin{lstlisting}[language=Python, caption={Example Python function for calculating factorial.}, label={lst:PythonExample}]
def factorial(n):
    """Calculate the factorial of n recursively."""
    if n <= 1:
        return 1
    else:
        return n * factorial(n - 1)

# Example usage
result = factorial(5)  # Returns 120
print(f"5! = {result}")
\end{lstlisting}

Reference code listings as Listing~\ref{lst:PythonExample}.

\subsection{Different Programming Languages}
\label{sec:HowTo_CodeLanguages}

The \texttt{listings} package supports many languages:

\begin{lstlisting}[language=C, caption={Example C code for bubble sort.}, label={lst:CExample}]
void bubbleSort(int arr[], int n) {
    for (int i = 0; i < n - 1; i++) {
        for (int j = 0; j < n - i - 1; j++) {
            if (arr[j] > arr[j + 1]) {
                // Swap elements
                int temp = arr[j];
                arr[j] = arr[j + 1];
                arr[j + 1] = temp;
            }
        }
    }
}
\end{lstlisting}

\begin{lstlisting}[language=Java, caption={Example Java class definition.}, label={lst:JavaExample}]
public class Person {
    private String name;
    private int age;
    
    public Person(String name, int age) {
        this.name = name;
        this.age = age;
    }
    
    public String getName() {
        return this.name;
    }
}
\end{lstlisting}

%=============================================================================
\section{Algorithms and Pseudocode}
\label{sec:HowTo_Algorithms}
%=============================================================================

For presenting algorithms, use the \texttt{algorithm} and \texttt{algpseudocode} packages:

\begin{algorithm}[htbp]
  \caption{Binary Search Algorithm}
  \label{alg:BinarySearch}
  \begin{algorithmic}[1]
    \Require Sorted array $A[1..n]$, target value $x$
    \Ensure Index of $x$ in $A$, or $-1$ if not found
    \State $left \gets 1$
    \State $right \gets n$
    \While{$left \leq right$}
    \State $mid \gets \lfloor (left + right) / 2 \rfloor$
    \If{$A[mid] = x$}
    \State \Return $mid$
    \ElsIf{$A[mid] < x$}
    \State $left \gets mid + 1$
    \Else
    \State $right \gets mid - 1$
    \EndIf
    \EndWhile
    \State \Return $-1$
  \end{algorithmic}
\end{algorithm}

Algorithm~\ref{alg:BinarySearch} shows the classic binary search with $O(\log n)$ time complexity.

%=============================================================================
\section{Citations and Bibliography}
\label{sec:HowTo_Citations}
%=============================================================================

\subsection{Citation Commands}
\label{sec:HowTo_CitationCommands}

This template uses BibLaTeX with the \texttt{natbib} compatibility layer. Common citation commands:

\begin{itemize}
  \item \texttt{\textbackslash citep\{key\}} -- Parenthetical citation: \citep{friese_handbuch_2020}
  \item \texttt{\textbackslash citet\{key\}} -- Textual citation: \citet{friese_handbuch_2020}
  \item \texttt{\textbackslash cite\{key\}} -- Basic citation: \cite{friese_handbuch_2020}
\end{itemize}

For multiple citations: \citep{friese_handbuch_2020, beermann_veranderungen_2020}.

\subsection{Managing Bibliography}
\label{sec:HowTo_Bibliography}

Bibliography entries are stored in \texttt{bibliography.bib}. Common entry types:

\begin{itemize}
  \item \texttt{@article} -- Journal articles
  \item \texttt{@book} -- Books
  \item \texttt{@inproceedings} -- Conference papers
  \item \texttt{@techreport} -- Technical reports
  \item \texttt{@misc} -- Websites and other sources
  \item \texttt{@phdthesis}, \texttt{@mastersthesis} -- Theses
\end{itemize}

%=============================================================================
\section{Acronyms and Glossary}
\label{sec:HowTo_Acronyms}
%=============================================================================

The \texttt{glossaries} package manages acronyms and glossary entries.

\subsection{Using Acronyms}
\label{sec:HowTo_UsingAcronyms}

Acronym commands:
\begin{itemize}
  \item \texttt{\textbackslash acrshort\{key\}} -- Short form: \acrshort{gcd}
  \item \texttt{\textbackslash acrlong\{key\}} -- Long form: \acrlong{gcd}
  \item \texttt{\textbackslash acrfull\{key\}} -- Full form: \acrfull{lcm}
  \item \texttt{\textbackslash gls\{key\}} -- Smart reference (expands on first use)
\end{itemize}

The \acrfull{gcd} is used in number theory to find common factors. Later references use \acrshort{gcd}.

\subsection{Glossary Entries}
\label{sec:HowTo_Glossary}

Glossary entries are referenced with \texttt{\textbackslash gls\{key\}}, \texttt{\textbackslash Gls\{key\}} (capitalized), or \texttt{\textbackslash glspl\{key\}} (plural). For example: \Gls{latex} is a document preparation system based on \gls{maths}.

%=============================================================================
\section{Lists and Enumerations}
\label{sec:HowTo_Lists}
%=============================================================================

\subsection{Itemized Lists}
\label{sec:HowTo_Itemize}

\begin{itemize}
  \item First level item
  \item Another first level item
        \begin{itemize}
          \item Second level item
          \item Another second level item
                \begin{itemize}
                  \item Third level item
                \end{itemize}
        \end{itemize}
\end{itemize}

\subsection{Enumerated Lists}
\label{sec:HowTo_Enumerate}

\begin{enumerate}
  \item First step
  \item Second step
        \begin{enumerate}
          \item Sub-step 2.1
          \item Sub-step 2.2
        \end{enumerate}
  \item Third step
\end{enumerate}

\subsection{Description Lists}
\label{sec:HowTo_Description}

\begin{description}
  \item[Term 1] Definition of the first term.
  \item[Term 2] Definition of the second term with more detailed explanation.
  \item[Term 3] Brief definition.
\end{description}

%=============================================================================
\section{Text Formatting}
\label{sec:HowTo_Formatting}
%=============================================================================

\subsection{Basic Text Styles}
\label{sec:HowTo_TextStyles}

\begin{itemize}
  \item \textbf{Bold text} -- \texttt{\textbackslash textbf\{...\}}
  \item \textit{Italic text} -- \texttt{\textbackslash textit\{...\}}
  \item \underline{Underlined text} -- \texttt{\textbackslash underline\{...\}}
  \item \texttt{Monospace text} -- \texttt{\textbackslash texttt\{...\}}
  \item \textsc{Small Caps} -- \texttt{\textbackslash textsc\{...\}}
  \item \sout{Strikethrough text} -- \texttt{\textbackslash sout\{...\}} (requires \texttt{ulem} package)
\end{itemize}

\subsection{Quotations}
\label{sec:HowTo_Quotations}

For short inline quotes, use quotation marks: ``This is a quote.''

For longer block quotes, use the \texttt{quote} or \texttt{quotation} environment:

\begin{quote}
  \colorbox{gray!15}{\parbox{\dimexpr\linewidth-2\fboxsep}{%
    This is a block quotation. It is indented from both margins and is useful for longer quoted passages from other sources.%
  }}
\end{quote}

%=============================================================================
\section{Comments and TODOs}
\label{sec:HowTo_Comments}
%=============================================================================

\subsection{LaTeX Comments}
\label{sec:HowTo_LaTeXComments}

Use \texttt{\%} for single-line comments. For multi-line comments, use the \texttt{comment} environment:

% This is a single-line comment
\begin{comment}
This entire block is commented out.
It will not appear in the output.
Useful for temporarily removing large sections.
\end{comment}

\subsection{TODO Notes}
\label{sec:HowTo_TodoNotes}

This template provides inline TODO notes for review:

\mycomment{This is an example TODO note that appears inline in the document.}

These notes are visible during writing but can be disabled for final submission.

%=============================================================================
\section{Hyperlinks and URLs}
\label{sec:HowTo_Hyperlinks}
%=============================================================================

The \texttt{hyperref} package enables clickable links:

\begin{itemize}
  \item URLs: \url{https://www.latex-project.org/}
  \item Named links: \href{https://www.overleaf.com}{Overleaf Online Editor}
  \item Email: \href{mailto:example@university.edu}{example@university.edu}
\end{itemize}

All cross-references, citations, and table of contents entries are automatically hyperlinked.

%=============================================================================
\section{Best Practices}
\label{sec:HowTo_BestPractices}
%=============================================================================

\begin{enumerate}
  \item \textbf{Use meaningful labels}: Prefix labels with the type (e.g., \texttt{fig:}, \texttt{tab:}, \texttt{eq:}, \texttt{sec:}, \texttt{lst:}, \texttt{alg:}).

  \item \textbf{Compile multiple times}: Run the compiler at least twice to resolve cross-references. For bibliography, run: \texttt{pdflatex} $\rightarrow$ \texttt{biber} $\rightarrow$ \texttt{pdflatex} $\rightarrow$ \texttt{pdflatex}.

  \item \textbf{Use non-breaking spaces}: Use \texttt{\~{}} before references to prevent line breaks: \texttt{Figure\~{}\textbackslash ref\{fig:example\}}.

  \item \textbf{Keep source files organized}: Use separate \texttt{.tex} files for each chapter/section.

  \item \textbf{Use vector graphics}: Prefer PDF or SVG for diagrams and figures when possible.

  \item \textbf{Backup your work}: Use version control (Git) and regular backups.

  \item \textbf{Check for overfull boxes}: Review warnings about overfull \texttt{hbox} or \texttt{vbox} and adjust text/figures accordingly.
\end{enumerate}
