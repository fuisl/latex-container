\chapter{Engineering Method}
\label{ch:EngineeringMethod}

% Page budget: 3 pages (Pages 8-10)
% This replaces the old "end-to-end system overview" with a method-first view that applies to both tracks

%=============================================================================
\section{R\&D Method: Exploration to Production}
\label{sec:RDMethod}
%=============================================================================

% 1.5 pages
% How you explore options
% How you validate (benchmarks, prototypes, operational signals)
% What counts as evidence
% Iterative cycle: exploration → evidence → decision → hardening

% Figure placeholder: F5 - R&D lifecycle diagram (Explore → Decide → Ship → Observe)
% \begin{figure}[htbp]
%   \centering
%   % \includegraphics[width=0.9\textwidth]{Figures/rd-lifecycle.pdf}
%   \caption{R\&D lifecycle showing the iterative flow from exploration through decision to production hardening.}
%   \label{fig:rd_lifecycle}
% \end{figure}


%=============================================================================
\section{Decision Documentation Practice (ADR Mini)}
\label{sec:ADRPractice}
%=============================================================================

% 1 page
% Standard template used throughout report:
% - Context
% - Options considered  
% - Criteria
% - Evidence
% - Decision
% - Trade-offs + mitigations
%
% Emphasize trade-offs + mitigations

% Table placeholder: T3 - Decision criteria (latency, cost, maintainability, correctness, security, operability)
% \begin{table}[htbp]
%   \centering
%   \caption{Decision criteria used across all engineering decisions.}
%   \label{tab:DecisionCriteria}
%   \begin{tabular}{lp{8cm}l}
%     \toprule
%     \textbf{Criterion} & \textbf{Definition} & \textbf{Weight} \\
%     \midrule
%     Latency & Response time for user-facing operations & High \\
%     Cost & Infrastructure and API costs & Medium \\
%     Maintainability & Ease of future changes and debugging & High \\
%     Correctness & Accuracy and reliability of outputs & Critical \\
%     Security & Protection of data and access control & Critical \\
%     Operability & Ease of monitoring, deployment, and recovery & High \\
%     \bottomrule
%   \end{tabular}
% \end{table}


%=============================================================================
\section{Workflow Systems vs Agent Systems}
\label{sec:WorkflowVsAgent}
%=============================================================================

% 0.5 page - HIGHLY IMPORTANT for differentiation
% This section shows maturity in understanding different system paradigms

% Workflow automation characteristics:
% - Deterministic orchestration
% - Integrations
% - Retries and idempotency
% - Audit logs

% Agent systems characteristics:
% - Stateful reasoning
% - Tool use
% - Uncertainty
% - Evaluation/guardrails
% - Why the architectures differ

% Figure placeholder: F4 - Workflow vs Agent comparison diagram (side-by-side)
% \begin{figure}[htbp]
%   \centering
%   % \includegraphics[width=\textwidth]{Figures/workflow-vs-agent.pdf}
%   \caption{Comparison of workflow automation systems and agent systems, highlighting key architectural differences.}
%   \label{fig:workflow_vs_agent}
% \end{figure}

