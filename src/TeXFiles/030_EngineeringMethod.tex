%=============================================================================
\chapter{Engineering Method}
\label{ch:EngineeringMethod}
%=============================================================================
% Page budget: ~3 pages (Pages 8--10)
%=============================================================================

This chapter describes the engineering methodology used throughout the
internship. The approach emphasizes structured exploration, evidence-based
decisions, and incremental hardening toward production readiness.
The same method was applied across workflow automation, infrastructure
experimentation, and agent platform development.

%=============================================================================
\section{R\&D Method: Exploration to Production}
\label{sec:RDMethod}
%=============================================================================

Engineering work followed an iterative \gls{rd} lifecycle designed to reduce
technical risk while maintaining delivery velocity. Each task moved through
four recurring phases (see Figure~\ref{fig:rd_lifecycle}):

\begin{itemize}
    \item \textbf{Exploration}: Objectives and desired outcomes were clarified,
          the ecosystem was surveyed, and official documentation was studied to
          understand usage patterns, constraints, and common failure modes
          \cite{googleTechWriting,diataxis}.

    \item \textbf{Prototyping and Evidence Gathering}: Minimal prototypes were
          built to validate feasibility, followed by lightweight benchmarking or
          operational observation to assess performance, cost, and correctness.

    \item \textbf{Decision and Commitment}: A primary approach was selected once
          baseline requirements were satisfied, enabling deeper iteration and
          architectural refinement.

    \item \textbf{Hardening and Validation}: Reliability, configurability, and
          operational safety were improved to reach an \gls{mvp} state suitable
          for production \cite{googleSRE,microsoftWellArchitected}.
\end{itemize}

\begin{figure}[htbp]
    \centering
    \includegraphics[width=0.5\textwidth]{Figures/rd_lifecycle_diagram.png}
    \caption{\gls{rd} lifecycle showing iterative progression from exploration
        through decision to production hardening.}
    \label{fig:rd_lifecycle}
\end{figure}

Exploration was time-bounded. Once an approach proved viable and aligned with
objectives, further comparison was deprioritized in favor of iterative
improvement.

Validation criteria were task-specific. Evidence used to validate solutions
included:
\begin{itemize}
    \item correctness of execution flow and state transitions,
    \item human evaluation of generated artifacts,
    \item runtime behavior in staging (latency, execution time, cost),
    \item review or approval by senior engineers.
\end{itemize}

A task was considered ready to progress when it met its definition of done,
reached an \gls{mvp} suitable for production, and passed peer or supervisor
review.


%=============================================================================
\section{Decision Documentation Practice (ADR Mini)}
\label{sec:ADRPractice}
%=============================================================================

Major engineering decisions were documented using a lightweight \gls{adr}
format \cite{nygard2011adr}. This was applied informally during development
and later formalized in pull request descriptions, code comments, and internal
documentation.

Each decision followed the same structure:
\begin{itemize}
    \item \textbf{Context}: The problem and constraints behind the decision.
    \item \textbf{Options}: Alternatives that were considered.
    \item \textbf{Criteria}: Factors used to compare options.
    \item \textbf{Evidence}: Prototypes, observations, or feedback.
    \item \textbf{Decision}: The selected approach.
    \item \textbf{Trade-offs}: Known limitations and how they are addressed.
\end{itemize}

This practice kept decisions focused on extensibility, operational
predictability, and simplicity over short-term convenience.

\begin{table}[htbp]
    \centering
    \captionsetup{font=normalsize}
    \footnotesize
    \caption{Decision criteria applied across engineering tasks.}
    \label{tab:DecisionCriteria}
    \begin{tabularx}{\textwidth}{|p{2.6cm}|X|p{1.8cm}|}
        \hline
        \textbf{Criterion} & \textbf{Definition}                                 & \textbf{Priority} \\
        \hline
        Extensibility      & Ability to evolve the system without major redesign & High              \\
        \hline
        Performance        & Latency and cost under realistic workloads          & High              \\
        \hline
        Maintainability    & Ease of debugging and future changes                & High              \\
        \hline
        Correctness        & Reliability and accuracy of system outputs          & Critical          \\
        \hline
        Security           & Protection of data and access boundaries            & Critical          \\
        \hline
        Operability        & Monitoring, deployment, and recovery capability     & High              \\
        \hline
    \end{tabularx}
\end{table}

%=============================================================================
\section{Workflow Systems vs Agent Systems}
\label{sec:WorkflowVsAgent}
%=============================================================================

A clear distinction was maintained between \textbf{workflow automation systems}
and \textbf{agent systems}. These paradigms address different problem classes
(see Figure~\ref{fig:workflow_vs_agent}) \cite{googleSRE,openaiPrompting}.

\textbf{Workflow systems} are characterized by:
\begin{itemize}
    \item deterministic orchestration with predefined execution paths,
    \item explicit retries and idempotency guarantees,
    \item strong auditability and traceability,
    \item suitability for integration and compliance tasks.
\end{itemize}

\textbf{Agent systems} emphasize:
\begin{itemize}
    \item stateful reasoning across multiple steps,
    \item tool invocation under uncertainty,
    \item non-deterministic behavior driven by \glspl{llm},
    \item the need for evaluation, guardrails, and governance.
\end{itemize}

These approaches exist on a spectrum rather than as a binary choice. As systems
become more agentic, flexibility increases---but so do risks around correctness,
observability, and control. This distinction informed architectural choices
and justified using different systems for different tasks.

\begin{figure}[htbp]
    \centering
    % Side-by-side diagram:
    % Left: Workflow system (boxes, arrows, retries, logs)
    % Right: Agent system (state container, decision nodes, tool calls, loops)
    \includegraphics[width=\textwidth]{Figures/workflow-agent-figure.png}
    \caption{Workflow automation vs agent systems: architectural and operational
        differences.}
    \label{fig:workflow_vs_agent}
\end{figure}