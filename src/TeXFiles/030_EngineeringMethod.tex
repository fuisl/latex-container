%=============================================================================
\chapter{Engineering Method}
\label{ch:EngineeringMethod}
%=============================================================================
% Page budget: ~3 pages (Pages 8--10)
%=============================================================================

This chapter describes the engineering methodology applied throughout the
internship. The method emphasizes structured exploration, evidence-based
decision making, and incremental hardening toward production readiness.
It was applied consistently across independent initiatives, including
workflow automation, infrastructure experimentation, and agent platform
development.

%=============================================================================
\section{R\&D Method: Exploration to Production}
\label{sec:RDMethod}
%=============================================================================

Engineering work during the internship followed an iterative
\gls{rd} lifecycle designed to reduce technical risk while
maintaining delivery velocity. Each task progressed through four recurring
phases as described in figure \ref{fig:rd_lifecycle}:

\begin{itemize}
    \item \textbf{Exploration}: Clarifying objectives and desired outcomes,
          surveying the ecosystem, and studying official documentation to understand
          intended usage patterns, constraints, and common failure modes
          \cite{googleTechWriting,diataxis}.

    \item \textbf{Prototyping and Evidence Gathering}: Implementing minimal,
          focused prototypes to validate feasibility, followed by lightweight
          benchmarking or operational observation to assess performance, cost, and
          correctness.

    \item \textbf{Decision and Commitment}: Selecting a primary approach once it
          satisfied baseline requirements, and committing to it early in order to
          enable deeper iteration and architectural refinement.

    \item \textbf{Hardening and Validation}: Improving reliability,
          configurability, and operational safety to reach a
          \gls{mvp} state suitable for production use
          \cite{googleSRE,microsoftWellArchitected}.
\end{itemize}

\begin{figure}[htbp]
    \centering
    \includegraphics[width=0.5\textwidth]{Figures/rd_lifecycle_diagram.png}
    \caption{\gls{rd} lifecycle illustrating the iterative progression from
        exploration through decision to production hardening.}
    \label{fig:rd_lifecycle}
\end{figure}

Exploration was intentionally time-bounded. Once an approach was deemed viable
and aligned with the task objective, further comparison was deprioritized in
favor of iterative improvement following an agile development model.

Validation criteria were task-specific rather than benchmark-driven. Evidence
used to validate solutions included:
\begin{itemize}
    \item correctness of execution flow and explicit state transitions,
    \item human evaluation of generated artifacts,
    \item runtime behavior observed in staging environments (e.g., latency,
          execution time, and cost),
    \item and review or approval by senior engineers.
\end{itemize}

A task was considered ready to progress when it satisfied its definition of
done, achieved an \gls{mvp} suitable for a production environment, and passed
peer or supervisor review.


%=============================================================================
\section{Decision Documentation Practice (ADR Mini)}
\label{sec:ADRPractice}
%=============================================================================

To maintain consistency and traceability, major engineering decisions were
evaluated using a lightweight \gls{adr} format
\cite{nygard2011adr}. This format was applied informally during development
and later formalized through pull request descriptions, code comments, and
internal documentation.

Each decision followed the same structure:
\begin{itemize}
    \item \textbf{Context}: The problem and constraints motivating the decision.
    \item \textbf{Options}: Viable alternatives considered.
    \item \textbf{Criteria}: Factors used to compare options.
    \item \textbf{Evidence}: Prototypes, observations, or feedback.
    \item \textbf{Decision}: The selected approach.
    \item \textbf{Trade-offs and Mitigations}: Known limitations and how they are
          addressed.
\end{itemize}

This practice ensured that decisions prioritized long-term extensibility,
operational predictability, and simplicity, rather than short-term
implementation convenience.

\begin{table}[htbp]
    \centering
    \captionsetup{font=normalsize}
    \footnotesize
    \caption{Decision criteria applied across engineering tasks.}
    \label{tab:DecisionCriteria}
    \begin{tabularx}{\textwidth}{|p{2.6cm}|X|p{1.8cm}|}
        \hline
        \textbf{Criterion} & \textbf{Definition}                                 & \textbf{Priority} \\
        \hline
        Extensibility      & Ability to evolve the system without major redesign & High              \\
        \hline
        Performance        & Latency and cost under realistic workloads          & High              \\
        \hline
        Maintainability    & Ease of debugging and future changes                & High              \\
        \hline
        Correctness        & Reliability and accuracy of system outputs          & Critical          \\
        \hline
        Security           & Protection of data and access boundaries            & Critical          \\
        \hline
        Operability        & Monitoring, deployment, and recovery capability     & High              \\
        \hline
    \end{tabularx}
\end{table}

%=============================================================================
\section{Workflow Systems vs Agent Systems}
\label{sec:WorkflowVsAgent}
%=============================================================================

A clear conceptual distinction was maintained between
\textbf{workflow automation systems} and \textbf{agent systems}, as the two
paradigms (figure \ref{fig:workflow_vs_agent}) address fundamentally different problem classes
\cite{googleSRE,openaiPrompting}.

\textbf{Workflow systems} are characterized by:
\begin{itemize}
    \item deterministic orchestration with predefined execution paths,
    \item explicit retries and idempotency guarantees,
    \item strong auditability and traceability,
    \item suitability for integration-heavy and compliance-sensitive tasks.
\end{itemize}

\textbf{Agent systems}, by contrast, emphasize:
\begin{itemize}
    \item stateful reasoning across multiple steps,
    \item tool invocation under uncertainty,
    \item non-deterministic behavior driven by \glspl{llm},
    \item the need for explicit evaluation, guardrails, and governance mechanisms.
\end{itemize}

Rather than a binary distinction, these approaches form a spectrum. As systems
become more agentic, flexibility increases, but so do risks related to
correctness, observability, and control. This distinction informed
architectural choices throughout the internship and justified the use of
different systems for different tasks.

\begin{figure}[htbp]
    \centering
    % Side-by-side diagram:
    % Left: Workflow system (boxes, arrows, retries, logs)
    % Right: Agent system (state container, decision nodes, tool calls, loops)
    \includegraphics[width=\textwidth]{Figures/workflow-agent-figure.png}
    \caption{Conceptual comparison of workflow automation systems and agent
        systems, highlighting architectural and operational differences.}
    \label{fig:workflow_vs_agent}
\end{figure}