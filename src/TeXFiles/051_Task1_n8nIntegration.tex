%=============================================================================
\chapter{ERP Integration via n8n}
\label{ch:Task1_n8n}
%=============================================================================
% Page budget: 5 pages (Pages 13--17)
%=============================================================================

This chapter describes the design and implementation of an \gls{erp}
integration using \gls{n8n} as a workflow automation
platform~\cite{n8nDocs}. The work introduces AI-assisted processing into
existing enterprise workflows while preserving deterministic execution,
operational safety, and auditability.

Unlike agent-based systems explored later in the internship, this integration
follows a workflow-oriented model. Control flow, branching, and failure
handling are explicitly defined, following established distinctions between
deterministic workflows and agent-based systems~\cite{anthropicAgents2024}.

%=============================================================================
\section{Problem Statement and Requirements}
\label{sec:Task1_Problem}
%=============================================================================

Enterprise Resource Planning systems operate at the core of business processes
and are highly sensitive to integration errors, unexpected behavior, or partial
failures. While AI-assisted capabilities can add significant value---such as
decision support or automated analysis---directly embedding non-deterministic
logic into ERP systems introduces unacceptable operational risk in production
environments~\cite{googleSRE}.

The problem addressed was to enable AI-assisted functionality within
ERP-related processes without compromising stability or predictability. The
integration needed to respond to ERP-triggered events, perform AI-assisted
computation when appropriate, and apply resulting actions back to the ERP
system in a controlled and observable manner.

The design was guided by explicit constraints:

\begin{itemize}
    \item \textbf{Deterministic execution:} workflow paths must be reproducible
          and understandable by operators.
    \item \textbf{Failure isolation:} errors in AI-assisted steps must not
          corrupt ERP state or block critical workflows.
    \item \textbf{Traceability:} all executions must be observable and auditable
          for debugging and review~\cite{googleSRE}.
    \item \textbf{Extensibility:} workflows should support future changes
          without requiring a complete redesign.
\end{itemize}

Advanced agent behaviors such as autonomous decision-making or cross-run
learning were considered out of scope. These concerns were deferred to later
work on agent-oriented architectures, where additional governance and
evaluation mechanisms are required~\cite{anthropicAgents2024}.

%=============================================================================
\section{Workflow Design}
\label{sec:Task1_WorkflowDesign}
%=============================================================================

The integration was implemented using n8n, which provides a visual and
programmatic model for deterministic workflow orchestration~\cite{n8nDocs}.
This choice enabled explicit control over execution order, branching logic,
and failure handling---capabilities essential when integrating with critical
enterprise systems.

The workflow is initiated by a well-defined trigger, such as an event emitted
by the ERP system or a scheduled invocation. Execution proceeds through a
sequence of nodes, each responsible for a single, well-scoped operation. Early
stages perform input validation and sanity checks to ensure downstream
processing operates on well-formed data.

AI-assisted processing is encapsulated within dedicated workflow nodes. These
nodes receive validated inputs, invoke external large language models via
well-defined APIs, and return structured outputs that downstream nodes can
interpret deterministically. Constraining AI interactions to bounded steps
avoids uncontrolled propagation of uncertainty and preserves predictable
behavior across executions.

Conditional branching routes execution based on both business rules and
AI-generated outputs. Each branch represents an explicitly modeled decision
path, making the workflow easier to reason about and review. This design
simplifies debugging, as operators can trace exactly which branch was taken
and why.

Failure handling is treated as a first-class concern. Transient failures, such
as network timeouts, are handled through retries with backoff, while
unrecoverable errors result in controlled abort paths. When partial side
effects may have occurred, compensation logic restores or neutralizes ERP state
before termination. The workflow also supports safe replay, allowing failed
executions to be retried after issues are resolved without introducing
duplicate effects~\cite{googleSRE}.

\begin{figure}[htbp]
    \centering
    % \includegraphics[width=\textwidth]{Figures/n8n-workflow-happy-path.pdf}
    \caption{n8n workflow diagram illustrating the main execution path for ERP
        integration.}
    \label{fig:n8n_workflow_happy}
\end{figure}

\begin{figure}[htbp]
    \centering
    % \includegraphics[width=\textwidth]{Figures/n8n-failure-branches.pdf}
    \caption{Failure handling branches showing retry, abort, and replay
        strategies used in the workflow.}
    \label{fig:n8n_failure_branches}
\end{figure}

\begin{figure}[htbp]
    \centering
    % \includegraphics[width=\textwidth]{Figures/erp-sequence-diagram.pdf}
    \caption{Sequence diagram showing the interaction between the ERP system
        and the n8n workflow during execution.}
    \label{fig:erp_sequence}
\end{figure}

%=============================================================================
\section{Operational Considerations}
\label{sec:Task1_Operations}
%=============================================================================

Operational robustness was a central design consideration. All secrets and
credentials are managed outside the workflow definition, allowing the same
workflow logic to be deployed across development, staging, and production
environments without modification. This separation aligns with best practices
in workflow automation and cloud-based integration
systems~\cite{googleSRE,microsoftWellArchitected}.

To support safe retries and replays, interactions with ERP systems are designed
to be idempotent wherever possible. Repeated execution of a workflow step does
not result in duplicate or inconsistent updates, even under failure
conditions~\cite{googleSRE}.

Rate limiting and throttling mechanisms are applied at integration boundaries
to protect both the ERP system and external AI services. These controls prevent
cascading failures during elevated load or abnormal behavior and provide
back-pressure when downstream systems are under stress~\cite{googleSRE}.

%=============================================================================
\section{Results and Learning Points}
\label{sec:Task1_Results}
%=============================================================================

The n8n-based integration demonstrated that AI-assisted processing can be
introduced into ERP workflows without sacrificing determinism or operational
safety. Explicit modeling of control flow and failure handling proved effective
for maintaining system stability and simplifying debugging.

The work also revealed limitations inherent to workflow-based approaches. As
AI-driven logic complexity increased, workflows became harder to maintain and
reason about. Expressing iterative or adaptive behavior required additional
branching and manual control structures, increasing maintenance overhead.

A key learning was the importance of separating deterministic workflow
automation from agent-based systems. Workflow tools are well suited for
control, safety, and auditability, but become strained when modeling adaptive
or iterative reasoning. This realization directly informed the architectural
direction taken in later work on agent-oriented
systems~\cite{anthropicAgents2024}.