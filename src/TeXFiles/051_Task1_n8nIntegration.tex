\chapter{ERP Integration via n8n}
\label{ch:Task1_n8n}

% Page budget: 5 pages (Pages 13-17)

%=============================================================================
\section{Problem Statement and Requirements}
\label{sec:Task1_Problem}
%=============================================================================

% 1 page
% What business problem needed solving
% Integration requirements and constraints



%=============================================================================
\section{Workflow Design}
\label{sec:Task1_WorkflowDesign}
%=============================================================================

% 2 pages
% Triggers, nodes, branching, retries, compensation logic
% Auditability and failure recovery patterns

% Figure placeholder: F7 - n8n workflow diagram (happy path)
% \begin{figure}[htbp]
%   \centering
%   % \includegraphics[width=\textwidth]{Figures/n8n-workflow-happy-path.pdf}
%   \caption{n8n workflow diagram showing the main happy path for ERP integration.}
%   \label{fig:n8n_workflow_happy}
% \end{figure}

% Figure placeholder: F8 - Failure handling branches (retry/abort/replay)
% \begin{figure}[htbp]
%   \centering
%   % \includegraphics[width=\textwidth]{Figures/n8n-failure-branches.pdf}
%   \caption{Failure handling branches showing retry, abort, and replay strategies.}
%   \label{fig:n8n_failure_branches}
% \end{figure}

% Figure placeholder: F9 - Sequence diagram (ERP event → n8n → action → ERP)
% \begin{figure}[htbp]
%   \centering
%   % \includegraphics[width=\textwidth]{Figures/erp-sequence-diagram.pdf}
%   \caption{Sequence diagram showing the flow from ERP event through n8n processing back to ERP.}
%   \label{fig:erp_sequence}
% \end{figure}


%=============================================================================
\section{Operational Considerations}
\label{sec:Task1_Operations}
%=============================================================================

% 1 page
% Secrets, environment separation, rate limiting, idempotency

% Table placeholder: T5 - Failure modes and mitigations
% \begin{table}[htbp]
%   \centering
%   \caption{Failure modes and corresponding mitigation strategies.}
%   \label{tab:Task1_FailureModes}
%   \begin{tabular}{lp{5cm}p{5cm}}
%     \toprule
%     \textbf{Failure Mode} & \textbf{Impact} & \textbf{Mitigation} \\
%     \midrule
%     Network timeout & ... & Retry with backoff \\
%     Invalid payload & ... & Validation + DLQ \\
%     Rate limit exceeded & ... & Queue with throttling \\
%     \bottomrule
%   \end{tabular}
% \end{table}


%=============================================================================
\section{Results and Learning Points}
\label{sec:Task1_Results}
%=============================================================================

% 1 page
% What worked, what didn't, and why

% Table placeholder: T4 - ADR: Why n8n, alternatives, trade-offs
% \begin{table}[htbp]
%   \centering
%   \caption{Architecture Decision Record: n8n selection.}
%   \label{tab:Task1_ADR}
%   \begin{tabular}{lp{10cm}}
%     \toprule
%     \textbf{Aspect} & \textbf{Details} \\
%     \midrule
%     Context & Need for workflow automation between ERP and AI services \\
%     Options & n8n, Temporal, custom scripts, Apache Airflow \\
%     Decision & n8n selected \\
%     Trade-offs & ... \\
%     \bottomrule
%   \end{tabular}
% \end{table}

% Learning Highlight Box:
% - What you assumed
% - What you observed  
% - What you changed
% - Principle for next time


