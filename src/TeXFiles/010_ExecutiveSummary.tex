\ifdefined\ThesisLanguageIsEnglish
    \chapter*{Executive Summary}
\else
    \chapter*{Kurzfassung}
\fi
\label{ch:ExecutiveSummary}
\addcontentsline{toc}{chapter}{Executive Summary}

% Page budget: 3 pages total (Pages 1--3)

%=============================================================================
% 0.1 Executive Summary (1 page)
%=============================================================================

This internship focused on the research, design, and implementation of
AI-assisted systems for enterprise software, with an emphasis on progressing
from exploratory experimentation to production-oriented engineering. The work
was structured around two independent but complementary engineering tracks:
\textbf{Track A: Workflow Automation and Infrastructure R\&D}, and
\textbf{Track B: LangGraph Agent Platform Engineering}.

Track A addressed the problem of safely integrating AI capabilities into
existing enterprise workflows. It included the design of a deterministic ERP
integration using workflow automation, evaluation of local and hosted model
serving strategies, deployment of a retrieval-augmented generation (RAG) agent
for product question answering, and exploratory research into tooling
interoperability via the Model Context Protocol (MCP). This track emphasized
risk isolation, auditability, and feasibility analysis, and served as a
foundation for understanding the operational constraints of AI systems in
enterprise environments.

Track B represented the core engineering contribution of the internship. It
focused on the design and implementation of an industrial-grade AI agent
platform using LangGraph. The platform supports quality assurance workflows by
transforming natural-language requirements into test cases, and test cases into
test steps, with explicit human-in-the-loop review at each critical decision
point. The system was implemented as a state-machine-driven service with
structured output contracts, governance controls for token usage, best-effort
observability via Langfuse, and a CI/CD pipeline targeting Google Cloud Run.

Key tangible outcomes of the internship include:

\begin{itemize}
    \item Potential ERP integration workflow using n8n with explicit
          failure handling and replay semantics.
    \item Evaluation and comparison of local model serving and enterprise-ready
          deployment approaches.
    \item Production-oriented LangGraph agent platform with human-in-the-loop
          execution, governance, and observability.
    \item CI/CD and deployment setup for operating the agent as a
          managed cloud service.
\end{itemize}

Across both tracks, the internship emphasized architectural decision-making,
trade-off analysis, and disciplined system design. The resulting work
demonstrates how AI capabilities can be integrated into enterprise software in a
controlled, observable, and maintainable manner, reflecting industry-grade
engineering practices rather than experimental prototypes.

%=============================================================================
% 0.2 Reading Guide and Structure Map (1 page)
%=============================================================================
\section*{Reading Guide}
\label{sec:ReadingGuide}

This report is organized to reflect the progression from exploratory research
to production-oriented system engineering. The two main tracks are presented as
\emph{independent engineering initiatives} that address different problem
classes and should be read accordingly.

\textbf{Track A} (Chapters 4--6) documents workflow automation, infrastructure
experiments, and protocol exploration. These chapters focus on feasibility,
constraints, and lessons learned from early-stage research. They do not attempt
to converge into a single system architecture.

\textbf{Track B} (Chapters 7--11) forms the industrial core of the report. These
chapters describe the design, implementation, governance, deployment, and
evaluation of the LangGraph-based agent platform. Readers primarily interested
in production architecture and agent engineering may focus on this track.

Later chapters synthesize insights across both tracks and reflect on technical
and professional growth achieved during the internship.

\begin{figure}[htbp]
    \centering
    \includegraphics[width=\textwidth]{Figures/report-structure-map.png}
    \caption{Report structure showing the two independent engineering tracks and
        their relationship to internship tasks.}
    \label{fig:report_structure_map}
\end{figure}

%=============================================================================
% 0.3 Traceability Matrix (1 page)
%=============================================================================
\section*{Traceability Matrix}
\label{sec:TraceabilityMatrix}

To ensure transparency and alignment with internship objectives, this section
maps the assigned tasks to concrete deliverables, report chapters, and the
skills developed during execution. This traceability highlights how each task
contributed to both technical outcomes and learning objectives.

\begin{table}[htbp]
    \centering
    \label{tab:DeliverablesOverview}
    \begin{tabular}{llp{10cm}l}
        \toprule
        \textbf{Track} & \textbf{Task} & \textbf{Deliverable}                                   & \textbf{Chapter} \\
        \midrule
        A              & Task 1        & ERP integration via workflow automation (n8n)          & Ch. 4            \\
        A              & Task 2        & Local and enterprise model serving evaluation          & Ch. 5            \\
        A              & Task 3        & RAG agent for product Q\&A                             & Ch. 5            \\
        A              & Task 1.5      & MCP exploration and interoperability research          & Ch. 6            \\
        B              & Task 4        & QA agent design using LangChain and LangGraph          & Ch. 8            \\
        B              & Task 5        & CI/CD pipeline and cloud deployment (GCP)              & Ch. 10           \\
        B              & Task 6--7     & Agent integration and production evaluation (Langfuse) & Ch. 9--11        \\
        \bottomrule
    \end{tabular}
    \caption{Overview of deliverables organized by engineering track.}
\end{table}