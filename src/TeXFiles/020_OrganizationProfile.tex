\chapter{Organization Profile and Internship Setup}
\label{ch:OrgProfile}

% Page budget: ~4 pages
% Covers guideline: Profile of organization, Tasks assigned by senior role, Intra-organization communication,
% Results/skills context (introduced here), and constraints shaping decisions.

%=============================================================================
\section{Organization Profile}
\label{sec:OrgProfile_Profile}
%=============================================================================

DevSamurai \cite{DevSamurai_Website} is a software engineering company that builds \textbf{business productivity applications and platform integrations} for major SaaS ecosystems. The company's products are most visible in the Atlassian ecosystem (Jira, Confluence) via the Atlassian Marketplace, with additional presence on monday.com, Salesforce AppExchange, and Microsoft AppSource. These ``workflow-native'' solutions embed into existing tools to improve productivity, collaboration, and operational visibility.

The company operates internationally, with headquarters in Japan and an engineering team in Vietnam. This geographic spread supports delivery at scale while providing access to specialized skills in product development, cloud infrastructure, and applied AI.

\subsection{Core Offerings and Technical Focus}
\label{subsec:OrgProfile_CoreOfferings}

DevSamurai's work is organized into two complementary streams:

\begin{enumerate}
    \item \textbf{Marketplace Applications and Integrations}
          \begin{itemize}
              \item Applications distributed through Atlassian Marketplace, Salesforce AppExchange, monday.com Marketplace, and Microsoft AppSource.
              \item Deep platform integration, with attention to permission models, API compatibility, and user experience that matches host products.
          \end{itemize}

    \item \textbf{Enterprise Engineering and Cloud/DevOps}
          \begin{itemize}
              \item Cloud-native practices that accelerate delivery and improve reliability (CI/CD pipelines, monitoring, secure deployment).
              \item Solutions for enterprise workflows: ERP-adjacent automation, data and analytics enablement, and applied AI for business process automation.
          \end{itemize}
\end{enumerate}

\subsection{Market Presence and Credibility}
\label{subsec:OrgProfile_Credibility}

DevSamurai targets global teams and displays customer adoption signals on its website. These signals are referenced here only to illustrate the company's market orientation and the production-grade standards applied to engineering work.

% Recommended figure placement: after introducing offerings and market orientation.
\begin{figure}[htbp]
    \centering
    \includegraphics[width=\textwidth]{Figures/devsamurai-trusted-by-logos.png}
    \caption{Customer adoption signals from the DevSamurai website (selected logos). This provides market context; internship deliverables were not shipped to these organizations.}
    \label{fig:devsamurai_trusted_by}
\end{figure}


%=============================================================================
\section{Role, Responsibilities, and Stakeholders}
\label{sec:OrgProfile_Role}
%=============================================================================

The internship was conducted within DevSamurai's R\&D and product engineering environment. Two primary goals were emphasized: (i) exploring solution approaches through targeted prototypes, and (ii) producing production-ready deliverables where operational maturity was required (deployment, governance, observability, and documentation).

\subsection{Stakeholders and Review Authority}
\label{subsec:OrgProfile_Stakeholders}

All work was reviewed by the following roles:

\begin{itemize}
    \item \textbf{Product Engineering Lead}: responsible for technical direction, architecture-level review, and alignment with product priorities.
    \item \textbf{Senior AI/ML Engineer}: reviewed evaluation methodology, agent and workflow design, and production hardening for AI systems.
    \item \textbf{DevOps/Reliability Engineer}: reviewed deployment topology, CI/CD quality gates, operational readiness, and failure handling.
\end{itemize}

\subsection{Responsibilities and Ownership}
\label{subsec:OrgProfile_Ownership}

Within agreed boundaries, responsibilities included:

\begin{itemize}
    \item Designing and implementing AI system prototypes (workflow automation, serving experiments, agent platform development).
    \item Documenting decisions in ADR-style records, supported by evidence such as benchmarks, evaluation results, and operational metrics.
    \item Delivering production-ready artifacts: code, configuration, infrastructure definitions, CI/CD pipelines, and deployment manifests.
    \item Producing operational documentation: runbooks, monitoring signal definitions, and deployment hardening checklists.
\end{itemize}

% Figure placeholder: stakeholder/communication diagram
% \begin{figure}[htbp]
%   \centering
%   % \includegraphics[width=0.85\textwidth]{Figures/stakeholder-diagram.pdf}
%   \caption{Stakeholder and communication model for the internship, including review points and acceptance gates.}
%   \label{fig:stakeholder_diagram}
% \end{figure}


%=============================================================================
\section{Communication and Execution Model}
\label{sec:OrgProfile_Communication}
%=============================================================================

An iterative execution model was followed to maintain stakeholder alignment and reduce technical risk early.

\subsection{Requirement Intake}
\label{subsec:OrgProfile_Intake}

Requirements and success criteria were established through intake discussions with product and engineering leadership. Each intake concluded with a scoped problem statement and acceptance criteria covering:

\begin{itemize}
    \item Scope boundaries and integration constraints
    \item Success metrics (quality targets, latency and cost budgets, operational readiness signals)
    \item Delivery timeline and dependency assumptions
    \item Security considerations (secrets handling, access control, logging constraints)
\end{itemize}

\subsection{Review and Iteration Cycles}
\label{subsec:OrgProfile_Iteration}

Work progressed through short iteration cycles with structured review points:

\begin{itemize}
    \item Regular check-ins with the Product Engineering Lead for scope alignment and architecture feedback.
    \item Technical reviews with the Senior AI/ML Engineer for design rationale, evaluation validity, and safety guardrails.
    \item Operational reviews with DevOps/Reliability before deployment, focusing on failure handling, observability, and rollback readiness.
\end{itemize}

\subsection{Handoff and Acceptance Criteria}
\label{subsec:OrgProfile_Acceptance}

A deliverable was considered ready for handoff when:

\begin{itemize}
    \item Architecture and key trade-offs were captured in decision records with supporting evidence.
    \item Code passed review and automated checks (tests, linting, reproducible builds).
    \item Operational documentation was complete and actionable (runbook, monitoring signals, hardening checklist).
    \item Deployment was reviewed and approved by DevOps.
\end{itemize}


%=============================================================================
\section{High-Level Challenges and Constraints}
\label{sec:OrgProfile_Challenges}
%=============================================================================

Technical priorities and design decisions were shaped by practical constraints common in production environments.

\subsection{Time Constraints}
\label{subsec:OrgProfile_TimeConstraints}

The internship was a fixed-duration engagement, which required disciplined scope control:

\begin{itemize}
    \item High-impact deliverables were prioritized over broad exploration.
    \item Risk drivers (reliability, cost, integration complexity) were identified early.
    \item Scope was kept tight to prevent uncontrolled expansion.
\end{itemize}

\subsection{Operational and Platform Constraints}
\label{subsec:OrgProfile_OperationalConstraints}

\begin{itemize}
    \item \textbf{Shared infrastructure}: Deployment targets (managed compute, Postgres, Redis) were shared resources, making safe rollout and isolation more important.
    \item \textbf{Data sensitivity}: Access to product and customer-adjacent data required careful logging and strict access controls.
    \item \textbf{External AI dependencies}: Third-party model providers introduced variability in latency, rate limits, and cost.
\end{itemize}

\subsection{Mitigation Strategies}
\label{subsec:OrgProfile_Mitigations}

The following mitigations were applied to reduce risk and maintain delivery quality:

\begin{table}[htbp]
    \centering
    \captionsetup{font=normalsize}
    \footnotesize
    \caption{Key constraints and mitigation strategies applied during the internship.}
    \label{tab:Constraints}
    \begin{tabularx}{\textwidth}{|p{2.5cm}|X|X|}
        \hline
        \textbf{Constraint}                       & \textbf{Primary Impact}                                             & \textbf{Mitigation Strategy}                                                             \\
        \hline
        Fixed internship timeline                 & Limited bandwidth for parallel exploration and production hardening & Phased delivery; explicit milestones; decision records to avoid rework                   \\
        \hline
        Shared production-adjacent infrastructure & Higher risk when testing and rolling out changes                    & Environment separation; feature flags; conservative rollout; runbook validation          \\
        \hline
        Third-party model providers               & Cost and latency variability; rate limiting                         & Cost budgets; usage tracking; retry/backoff policies; provider failure handling          \\
        \hline
        Data sensitivity                          & Compliance and privacy risk if mishandled in logs or traces         & Minimize sensitive payloads; redact identifiers; access control reviews; least-privilege \\
        \hline
    \end{tabularx}
\end{table}