\chapter{Research --- MCP Exploration}
\label{ch:MCPExploration}

% Page budget: 4 pages (Pages 23-26)
% MCP is explicitly placed here as part of the exploration phase, NOT as a LangGraph dependency

%=============================================================================
\section{Why MCP Was Explored}
\label{sec:MCP_Why}
%=============================================================================

% 0.5-1 page
% The problem: standardizing tool/context exposure across systems
% Research motivation



%=============================================================================
\section{MCP Conceptual Model}
\label{sec:MCP_Conceptual}
%=============================================================================

% 1 page
% Tools/resources/prompts concept
% Server/client boundaries
% What "interoperability" means in your context

% Figure placeholder: F13 - MCP conceptual diagram (client ↔ MCP server ↔ tools/resources)
% \begin{figure}[htbp]
%   \centering
%   % \includegraphics[width=\textwidth]{Figures/mcp-conceptual.pdf}
%   \caption{MCP conceptual model showing the relationship between clients, MCP servers, and exposed tools/resources.}
%   \label{fig:mcp_conceptual}
% \end{figure}


%=============================================================================
\section{Prototype Architecture and Experiments}
\label{sec:MCP_Prototype}
%=============================================================================

% 1 page
% What you built/tested
% Limitations found

% Figure placeholder: F14 - MCP prototype deployment (high-level)
% \begin{figure}[htbp]
%   \centering
%   % \includegraphics[width=0.8\textwidth]{Figures/mcp-prototype.pdf}
%   \caption{High-level deployment architecture of the MCP prototype.}
%   \label{fig:mcp_prototype}
% \end{figure}


%=============================================================================
\section{Outcomes and Lessons}
\label{sec:MCP_Outcomes}
%=============================================================================

% 1 page
% What MCP exploration taught you about interfaces, contracts, and tooling

% Table placeholder: T7 - Findings table (assumption → experiment → result → lesson)
% \begin{table}[htbp]
%   \centering
%   \caption{MCP exploration findings: from assumption to lesson learned.}
%   \label{tab:MCP_Findings}
%   \begin{tabular}{lp{3cm}p{3cm}p{4cm}}
%     \toprule
%     \textbf{Assumption} & \textbf{Experiment} & \textbf{Result} & \textbf{Lesson} \\
%     \midrule
%     ... & ... & ... & ... \\
%     ... & ... & ... & ... \\
%     \bottomrule
%   \end{tabular}
% \end{table}

% Learning Highlight Box:
% - What you assumed about tool interoperability
% - What you observed in MCP implementation
% - What you changed in your understanding
% - Principle for future tool integration work
