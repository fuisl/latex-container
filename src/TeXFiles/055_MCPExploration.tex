%=============================================================================
\chapter{Research --- MCP Exploration}
\label{ch:MCPExploration}
%=============================================================================
% Page budget: 4 pages (Pages 23--26)
% MCP is explicitly placed here as part of the exploration phase, NOT as a LangGraph dependency
%=============================================================================

This chapter documents the exploratory research conducted on the Model Context
Protocol (MCP). The purpose of this work was to evaluate MCP as a potential
approach for standardizing how language models interact with external tools,
resources, and application context. This exploration was conducted as part of
the early research phase of the internship and informed later architectural
decisions, but MCP itself was not adopted as a core dependency in subsequent
LangGraph-based development.

%=============================================================================
\section{Why MCP Was Explored}
\label{sec:MCP_Why}
%=============================================================================

As AI-assisted systems become more integrated into production applications, a
recurring challenge is how to expose tools, data sources, and application
context to language models in a consistent and reusable manner. Early agent
prototypes often rely on tightly coupled, framework-specific integrations,
making it difficult to reuse tooling across models or platforms.

The Model Context Protocol was explored as a response to this problem.
Introduced as an open protocol, MCP aims to standardize how models discover and
invoke tools, access structured resources, and receive contextual information
from external systems~\cite{anthropicMCP}. In principle, this abstraction
promises improved interoperability between models, tooling, and host
applications.

From a research perspective, MCP aligned well with the internship’s early
objectives: understanding interface boundaries, experimenting with
model-to-system contracts, and evaluating whether protocol-level abstractions
could simplify agent integration across heterogeneous environments. This made
MCP a suitable candidate for exploratory evaluation, even though its maturity
for production use remained uncertain.

%=============================================================================
\section{MCP Conceptual Model}
\label{sec:MCP_Conceptual}
%=============================================================================

At a conceptual level, MCP defines a clear separation between model clients and
external systems through an explicit server boundary. Rather than embedding
tool logic directly into an agent or framework, an MCP server advertises a set
of capabilities that models can discover and invoke dynamically.

The protocol distinguishes between three primary elements:

\begin{itemize}
    \item \textbf{Tools}, which represent executable actions exposed to the
          model through a well-defined interface.
    \item \textbf{Resources}, which provide structured or semi-structured data
          that models can query or retrieve.
    \item \textbf{Context}, which supplies metadata and environmental
          information required to ground model behavior.
\end{itemize}

In this model, the client (typically an LLM runtime or agent framework) does
not require prior knowledge of application-specific APIs. Instead, it relies
on the MCP server to describe available capabilities at runtime. This approach
shifts integration complexity away from the model and toward a protocol-driven
interface layer.

Within the context of this research, interoperability was understood as the
ability to reuse the same tool and resource definitions across different model
providers and agent implementations without rewriting integration logic. MCP
provided a concrete mechanism to evaluate whether such decoupling could be
achieved in practice.

\begin{figure}[htbp]
    \centering
    % \includegraphics[width=\textwidth]{Figures/mcp-conceptual.pdf}
    \caption{MCP conceptual model showing the relationship between clients, MCP
        servers, and exposed tools and resources.}
    \label{fig:mcp_conceptual}
\end{figure}

%=============================================================================
\section{Prototype Architecture and Experiments}
\label{sec:MCP_Prototype}
%=============================================================================

To evaluate MCP in practice, a lightweight prototype was implemented. The
prototype MCP server exposed a limited set of tools and resources backed by
existing application APIs. The goal was not feature completeness, but rather
to understand developer ergonomics, protocol clarity, and runtime behavior.

The experimental setup consisted of:
\begin{itemize}
    \item An MCP server advertising a small number of application-specific
          tools.
    \item A client-side agent invoking these tools through MCP abstractions.
    \item Manual inspection of tool invocation behavior and response handling.
\end{itemize}

While the prototype successfully demonstrated basic interoperability, several
limitations became apparent. Tool invocation semantics were heavily dependent
on model behavior, making outcomes sensitive to prompt phrasing and model
capabilities. Additionally, the lack of strong, enforceable contracts at the
protocol level made it difficult to guarantee correct or safe tool usage in
non-trivial scenarios.

Security considerations also emerged as a significant concern. Recent research
has shown that minimally scoped MCP servers can enable unintended data
exfiltration or cross-tool abuse if not carefully designed~\cite{croce2025trojans,song2025attack}.
These findings reinforced the importance of strong governance and validation
mechanisms, particularly in production environments.

\begin{figure}[htbp]
    \centering
    % \includegraphics[width=0.8\textwidth]{Figures/mcp-prototype.pdf}
    \caption{High-level deployment architecture of the MCP prototype used for
        experimentation.}
    \label{fig:mcp_prototype}
\end{figure}

%=============================================================================
\section{Outcomes and Lessons}
\label{sec:MCP_Outcomes}
%=============================================================================

The MCP exploration yielded several insights that influenced subsequent design
decisions. While the protocol offers an appealing abstraction for tool
interoperability, its practical application revealed trade-offs between
flexibility, safety, and determinism.

\begin{table}[htbp]
    \centering
    \caption{MCP exploration findings: from assumption to lesson learned.}
    \label{tab:MCP_Findings}
    \begin{tabular}{lp{3cm}p{3cm}p{4cm}}
        \toprule
        \textbf{Assumption}                            & \textbf{Experiment} & \textbf{Result} & \textbf{Lesson} \\
        \midrule
        Tool interfaces can be fully standardized      &
        Implement MCP tool exposure                    &
        Tool usage varied by model behavior            &
        Protocols need stronger behavioral constraints                                                           \\
        Interoperability reduces integration effort    &
        Reuse tools across agents                      &
        Integration simplified but debugging increased &
        Abstractions shift, not eliminate, complexity                                                            \\
        MCP is production-ready                        &
        Security review of prototype                   &
        Potential attack vectors identified            &
        Governance is mandatory for safe adoption                                                                \\
        \bottomrule
    \end{tabular}
\end{table}

A key outcome of this research was a refined understanding of interface design
for AI systems. Protocol-level abstractions can reduce coupling and improve
reuse, but they do not remove the need for explicit control over execution
flow, validation, and safety. These observations motivated a later shift
toward more structured orchestration approaches, where state transitions and
execution paths are explicitly modeled.

Overall, the MCP exploration served its intended purpose as a research probe.
It clarified the limits of protocol-driven interoperability and provided a
useful contrast to later work on stateful, workflow-oriented agent systems.