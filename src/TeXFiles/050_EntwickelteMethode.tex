% =============================================================================
% TRACK B: LangGraph Agent Platform Engineering
% Page budget: 22 pages (Pages 27-48) - but split across multiple files
% This file covers: Industrial Architecture (7 pages) + Graph Design (7 pages) = 14 pages
% =============================================================================

\chapter{LangGraph Industrial Architecture}
\label{ch:LangGraphArchitecture}

% Page budget: 7 pages (Pages 30-36)
% This is the centerpiece for professionalism

%=============================================================================
\section{Component Architecture}
\label{sec:ComponentArchitecture}
%=============================================================================

% 2 pages
% - LangGraph API service
% - Redis governance
% - Postgres checkpoints
% - LLM provider
% - Langfuse observability
% - Clients/consumers

% Figure placeholder: F17 - Component diagram (LangGraph platform)
% \begin{figure}[htbp]
%   \centering
%   % \includegraphics[width=\textwidth]{Figures/langgraph-component-diagram.pdf}
%   \caption{Component diagram showing the LangGraph platform architecture with all major services.}
%   \label{fig:langgraph_component}
% \end{figure}


%=============================================================================
\section{Deployment Architecture (Cloud Run)}
\label{sec:DeploymentArchitecture}
%=============================================================================

% 2 pages
% Service boundaries: API service + cron service
% Scaling model, concurrency, cold starts considerations
% Network/secrets policy

% Figure placeholder: F18 - Deployment topology (Cloud Run + dependencies)
% \begin{figure}[htbp]
%   \centering
%   % \includegraphics[width=\textwidth]{Figures/deployment-topology-cloudrun.pdf}
%   \caption{Deployment topology on GCP Cloud Run showing service boundaries and dependencies.}
%   \label{fig:deployment_topology}
% \end{figure}


%=============================================================================
\section{Runtime Request Lifecycle}
\label{sec:RuntimeLifecycle}
%=============================================================================

% 2 pages
% Invoke, interrupts, resume, checkpoints, callbacks

% Figure placeholder: F19 - Sequence diagram (invoke/resume lifecycle)
% \begin{figure}[htbp]
%   \centering
%   % \includegraphics[width=\textwidth]{Figures/invoke-resume-sequence.pdf}
%   \caption{Sequence diagram showing the invoke/resume lifecycle with interrupt and checkpoint handling.}
%   \label{fig:invoke_resume_sequence}
% \end{figure}

% Figure placeholder: F20 - "Control points" diagram (where governance/validation happens)
% \begin{figure}[htbp]
%   \centering
%   % \includegraphics[width=0.9\textwidth]{Figures/control-points.pdf}
%   \caption{Control points diagram showing where governance and validation occur in the request flow.}
%   \label{fig:control_points}
% \end{figure}


%=============================================================================
\section{Key Architectural Decisions}
\label{sec:KeyArchDecisions}
%=============================================================================

% 1 page, ADR summary
% State machines, HITL interrupts, structured outputs, best-effort tracing, pooling

% Table placeholder: T9 - ADR summary table (decision → options → trade-offs)
% \begin{table}[htbp]
%   \centering
%   \caption{Summary of key architectural decisions.}
%   \label{tab:ADRSummary}
%   \begin{tabular}{lp{4cm}p{5cm}}
%     \toprule
%     \textbf{Decision} & \textbf{Options Considered} & \textbf{Trade-offs} \\
%     \midrule
%     State machines & Custom, LangGraph & LangGraph: built-in persistence \\
%     HITL interrupts & Polling, Interrupts & Interrupts: cleaner UX \\
%     Structured outputs & Prompt, Schema & Schema: validation + reliability \\
%     Tracing & Strict, Best-effort & Best-effort: no blocking \\
%     \bottomrule
%   \end{tabular}
% \end{table}


%=============================================================================
%=============================================================================
\chapter{Graph Design and Runtime Behavior}
\label{ch:GraphDesign}

% Page budget: 7 pages (Pages 37-43)
% This section "differentiates agent from workflow" by showing state machines and formal contracts

%=============================================================================
\section{State and Schema Design}
\label{sec:StateSchemaDesign}
%=============================================================================

% 2 pages
% State schema, reducers
% Structured output contracts (with_structured_output)
% Validation strategy and failure recovery

% Figure placeholder: F21 - State schema + reducers diagram (state ownership)
% \begin{figure}[htbp]
%   \centering
%   % \includegraphics[width=0.9\textwidth]{Figures/state-schema-reducers.pdf}
%   \caption{State schema and reducers diagram showing state ownership and update patterns.}
%   \label{fig:state_schema}
% \end{figure}

% Figure placeholder: F22 - Structured output boundary diagram (schema contracts)
% \begin{figure}[htbp]
%   \centering
%   % \includegraphics[width=0.8\textwidth]{Figures/structured-output-boundary.pdf}
%   \caption{Structured output contract boundaries showing schema validation points.}
%   \label{fig:structured_output}
% \end{figure}


%=============================================================================
\section{Graph A: Requirement to Test Cases}
\label{sec:GraphA}
%=============================================================================

% 2 pages
% Transitions, regen modes, HITL gate, loop

% Figure placeholder: F23 - State machine diagram: Test Cases graph
% \begin{figure}[htbp]
%   \centering
%   % \includegraphics[width=\textwidth]{Figures/graph-a-state-machine.pdf}
%   \caption{State machine diagram for Graph A: Requirement → Test Cases.}
%   \label{fig:graph_a_state_machine}
% \end{figure}

% Table placeholder: T10 - Action → transition → node → state delta (Graph A)
% \begin{table}[htbp]
%   \centering
%   \caption{Graph A state transitions and their effects.}
%   \label{tab:GraphATransitions}
%   \begin{tabular}{llll}
%     \toprule
%     \textbf{Action} & \textbf{Transition} & \textbf{Node} & \textbf{State Delta} \\
%     \midrule
%     Generate & START → generate & generate\_cases & cases created \\
%     Review & generate → review & review\_gate & awaiting input \\
%     Approve & review → END & finalize & output ready \\
%     Reject & review → generate & regenerate & cases cleared \\
%     \bottomrule
%   \end{tabular}
% \end{table}


%=============================================================================
\section{Graph B: Test Case to Test Steps}
\label{sec:GraphB}
%=============================================================================

% 2 pages
% Edit semantics, reject + feedback + regen loop

% Figure placeholder: F24 - State machine diagram: Test Steps graph
% \begin{figure}[htbp]
%   \centering
%   % \includegraphics[width=\textwidth]{Figures/graph-b-state-machine.pdf}
%   \caption{State machine diagram for Graph B: Test Case → Test Steps.}
%   \label{fig:graph_b_state_machine}
% \end{figure}

% Table placeholder: T11 - Edit commands → validation → effect (Graph B)
% \begin{table}[htbp]
%   \centering
%   \caption{Edit command semantics and validation for Graph B.}
%   \label{tab:GraphBEditCommands}
%   \begin{tabular}{llp{5cm}}
%     \toprule
%     \textbf{Command} & \textbf{Validation} & \textbf{Effect} \\
%     \midrule
%     ADD & Step schema valid & Insert at position \\
%     DELETE & Step ID exists & Remove from list \\
%     UPDATE & Step ID exists, schema valid & Replace in place \\
%     \bottomrule
%   \end{tabular}
% \end{table}

% Table placeholder: T12 - Failure modes and recovery (schema/LLM/provider)
% \begin{table}[htbp]
%   \centering
%   \caption{Failure modes and recovery strategies.}
%   \label{tab:FailureRecovery}
%   \begin{tabular}{llp{5cm}}
%     \toprule
%     \textbf{Failure Type} & \textbf{Source} & \textbf{Recovery} \\
%     \midrule
%     Schema validation & LLM output & Retry with stricter prompt \\
%     Provider timeout & API & Retry with backoff \\
%     State corruption & Bug & Rollback to checkpoint \\
%     \bottomrule
%   \end{tabular}
% \end{table}


%=============================================================================
\section{HITL Interrupt and Resume Lifecycle}
\label{sec:HITLLifecycle}
%=============================================================================

% 1 page
% What is stored, what is validated, thread lifecycle

% Figure placeholder: F25 - Interrupt/resume timeline + checkpoint diagram
% \begin{figure}[htbp]
%   \centering
%   % \includegraphics[width=\textwidth]{Figures/hitl-lifecycle-timeline.pdf}
%   \caption{Timeline showing the HITL interrupt/resume lifecycle with checkpoint storage.}
%   \label{fig:hitl_timeline}
% \end{figure}
