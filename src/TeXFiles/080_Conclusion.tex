%=============================================================================
\chapter{Conclusion}
\label{ch:Conclusion}
%=============================================================================
% Page budget: ~0.5 page (Page 50)
%=============================================================================

\section{Summary}
\label{sec:Summary}

This internship focused on the research, design, and implementation of
AI-assisted systems across multiple levels of abstraction, ranging from
workflow automation and infrastructure exploration to a production-oriented
agent platform built using LangGraph. The work progressed deliberately from
exploratory prototypes toward an industrial-grade architecture, emphasizing
deterministic execution, human-in-the-loop control, and operational robustness.

Key outcomes included the successful integration of AI-assisted logic into
enterprise workflows using deterministic orchestration, the evaluation of
local and hosted model serving strategies, and the design and deployment of a
stateful agent platform capable of supporting iterative QA workflows. Through
the use of explicit state machines, structured outputs, governance controls,
and observability tooling, the final system demonstrates that AI agents can be
integrated into real engineering processes without sacrificing control or
reliability.

\section{Future Work and Roadmap}
\label{sec:FutureWork}

Several directions for future work naturally emerge from this internship.
From a technical perspective, further improvements could be made in automated
evaluation methodologies, particularly for agent-driven QA tasks where
standard benchmarks remain limited. Additional policy and safety mechanisms
could also be explored, such as stricter validation of resume payloads and more
granular governance controls.

From a platform perspective, the agent architecture can be extended to support
additional graphs, broader testing scenarios, and deeper integration with
existing development tooling. As model capabilities and tooling ecosystems
continue to evolve, the modular design of the system allows these advances to
be incorporated incrementally without requiring fundamental redesign.

\section{Personal Reflection}
\label{sec:PersonalReflection}

This internship provided valuable experience in bridging the gap between
research-oriented experimentation and production-quality engineering. Beyond
technical implementation, it emphasized the importance of architectural
discipline, clear decision-making criteria, and explicit handling of failure
modes in AI systems.

Working on real-world constraints such as cost governance, deployment pipelines,
and human-centered evaluation significantly shaped my understanding of how AI
systems should be built and operated in practice. The experience strengthened
my ability to reason about trade-offs, document design decisions, and approach
AI development as a systems engineering problem rather than a purely model-
centric one. These lessons will strongly inform my future academic and
professional work.